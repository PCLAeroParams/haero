\section{Proposed Input Specification}
\labelsection{input}

Below is a proposed input specification for MAM. The spec uses Yet Another
Markup Language (YAML). Compared to Fortran namelists, YAML offers

\begin{itemize}
  \item {\bf better readability}: YAML files are flexible, and don't have a lot
    of "kibble" (braces, tags, and other stuff you see routinely in fancier
    markup like XML, HTML, and so on). They're very easy to read.
  \item {\bf better validation}: YAML files have named entries that can be
    used or ignored, depending on the needs of the particular application.
    Also, the file format allows for error checking.
  \item {\bf better language support}: Fortran namelists are so-called because
    they are only available to Fortran. YAML has libraries that allows it to
    be used with several programming languages (including Fortran!). This makes
    it easier for tools and other applications to use similar input files. In
    particular, this allows workflows to use a single input file to define a
    workload that can be processed with several tools.
  \item {\bf support for lists and associative arrays}: YAML offers
    constructions for dynamically-sized datasets. As we move MAM toward runtime
    configurability, the ability to add species (and perhaps even modes) to an
    input file will become more important.
  \item {\bf a larger support community}: Fortran remains a niche language,
    which can instill a sense of pride in scientists. Unfortunately, the
    downside to belonging to a small community of specialists is that the tools
    are invariably of lower quality than more commonly-used tools. One needs
    only to witness the notorious ongoing issues with Fortran compilers to see
    this phenomenon firsthand.
\end{itemize}

The YAML spec can be used both for initializing data via the API, and with
standalone drivers. We also provide a proposed API for reading data from
input files.

\subsection*{Sections}

A YAML file consists of several named sections. Each of these sections can
contain data and metadata. Sections are a powerful tool for organizing input
using simple concepts with human-readable notation.

\subsubsection*{Modes}

The \verb modes  section defines the particle size modes available to a MAM
model. As we discussed in \refsection{modes_and_species}, a mode has metadata
specifying its size range and its geometric standard deviation.

\begin{verbatim}
modes:
- name: Aitken
  D_min: 0.0087
  D_max: 0.052
  sigma: 1.6
- name: Accumulation
  D_min: 0.0535
  D_max: 0.44
  sigma: 1.8
- name: Coarse
  D_min: 1.0
  D_max: 4.0
  sigma: 1.8
- name: Primary carbon
  D_min: 0.01
  D_max: 0.1
  sigma: 1.6
\end{verbatim}

Particle diameters and $\sigma$ are measured in $\mu\mathrm{m}$.

The \verb modes  section is essentially a list of named modes. In YAML, a list
contains items identified by hyphens. As you can see above, the definition of
each mode starts with its name, which is followed by its metadata. Modes can
overlap. Each mode must be completely specified---there are no default values.

Note that names don't require quotes around them.

\subsubsection*{Species}

Particle species exist within modes. As we discussed in
\refsection{modes_and_species}, a species is defined by its elemental
composition and its electric charge (given in units of the electronic charge
$|e|$).

\begin{verbatim}
species:
- name: SO2
  composition: {S: 1, O: 2}
  thermo:
  equation-of-state: ideal-gas
\end{verbatim}

The \verb species  section follows the same format as defined by
\href{https://cantera.org/documentation/dev/sphinx/html/yaml/species.html}{Cantera}.

\subsubsection*{Reactions}

Chemical reactions are defined in the \verb reactions  section.

This section follows the same format as defined by \href{https://cantera.org/documentation/dev/sphinx/html/yaml/reactions.html}{Cantera}.
The most interesting types of reactions for MAM are \verb elementary  reactions,
which use Arrhenius coefficients, and \verb falloff  reactions, which include
the \href{https://cantera.org/science/reactions.html#the-troe-falloff-function}{Troe falloff function}.

\subsubsection*{Initial Conditions}

The \verb initial-conditions section defines the initial state of an aerosol
system in MAM. These include number concentrations for species.

\begin{verbatim}
initial-conditions:
  SO2:
    concentration: 0.01
\end{verbatim}

This section is a mapping from species names to their initial concentrations.

\subsubsection*{Simulation Parameters}

\subsubsection*{Time Stepping}

\subsubsection*{Output}

\subsection{A Complete Example}

Let's see how we might define a MAM simulation for the standalone driver.

\begin{verbatim}
# Example MAM input specification (hashes are comments in YAML files).

\end{verbatim}

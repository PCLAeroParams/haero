\chapter{The Haero Library}
\labelchapter{library}

In this chapter, we describe the data structures and algorithms provided by the
Haero software library. We begin by discussing the data structures that
represent the important elements for any modal aerosol model. In particular,
the interface defines representations for:

\begin{itemize}
  \item {\bf Modes}: statistically-informed groupings of aerosol particles by
        size
  \item {\bf Species}: aerosol and gas molecules of interest. Each aerosol
        species belongs to a single aerosol mode and is tracked by mass and
        number.
  \item {\bf Chemical Reactions}: reactions within the chemical mechanism that
        governs how aerosol particles interact with each other, with the
        atmosphere, and with solar radation.
\end{itemize}

These entities represent the quantities of interest in Haero, and provide
{\bf metadata} needed to perform calculations on {\bf state data}, which is
stored in multi-dimensional arrays of real-valued numbers.

The modes and species present in the atmosphere don't uniquely specify the
parameters of a simulation, however. These species interact chemically with each
other, and with other species and also with radiation. Additionally, there are
numerical parameters that define grid parameters, choices of algorithms, and
other non-physical considerations. This information lives within an object
called a {\bf context}. There's one such context for each simulation, and Haero
uses the context to make decisions about how to numerically compute the
evolution of aerosols, and their effects on a system. The simulation context and
its interface is explained in \refsection{context}.

When a context has been created, one can initialize the simulation {\bf state}---
the set of quantities that define the physical state of an aerosol system. The
state consists of a set of multidimensional arrays that contain all relevant
physical quantities, and is defined by the \verb State  type described in
\refsection{state}

Given a system of modes, species, chemical reactions, etc, as represented by a
context, and given an initial state, Haero's application programming interface
(API) provides functions (subroutines) that evaluate aerosol
{\bf parametrizations}---the simplified models of the various physical processes
mentioned in \refchapter{physics}. Functions and subroutines are provided for
the C++ and Fortran programming languages. These functions are described in
detail in \refsection{parametrizations}, and listed comprehensively in
\refappendix{api}.

\section{Modes}\labelsection{modes}

We've seen how the dynamics of aerosols can be represented mathematically
by evolution equations for moments of modal distribution functions. Modes
simplify the description of aerosol particles in terms of their size: instead of
representing a population of particles with a distribution function
$n(V_p, \vec{x}, t)$ that varies continuously with the size of the particle, we
introduced $M$ discrete modes and declared that these modes partition the
population of aerosol particles in the sense of the modal assumption as given
by \refeq{modal_n}.

The essential information in a mode is the range of particle sizes it
encompasses, $[D_{\min}, D_{\max}]$, and its geometric standard deviation,
$\sigma_g$. In Haero's C and C++ interfaces,
we represent an aerosol mode with the \verb|Mode| struct:

\begin{verbatim}
struct Mode {
  std::string name;  // a unique identifier for the mode
  Real min_diameter; // the mode's minimum particle diameter
  Real max_diameter; // the mode's maximum particle diameter
  Real mean_std_dev; // the geometric mean standard deviation for the mode
};
\end{verbatim}

In principle, a Haero calculation can support any number of modes, but care
must be taken to ensure that the modal assumptions remain valid, and that
the parametrizations selected can accommodate the given modes.

\section{Species}\labelsection{species}

{\bf WARNING: This section is under construction!}

Each aerosol mode consists of one or more particle species. Additionally,
gas particles also come in different species. Both aerosol particles and
gas particles have physical properties that are described by the
\verb Species  data type.

A particle species is a specifically-identified molecular assembly with
a number of relevant physical properties. The fundamental description of a
species includes

\begin{itemize}
  \item a symbolic name (e.g. \verb SO4 , for sulfate)
  \item a descriptive name (e.g. \verb "sulfate" )
  \item its elemental composition (how many atoms of each type it possesses)
  \item its net electrical charge (in units of the elementary charge $|e|$)
\end{itemize}

We represent this information in the following way:

\begin{verbatim}
struct Species {
  std::string name;                       // full species name
  std::string symbol;                     // abbreviated symbolic name
  std::map<std::string, int> composition; // elemental make-up of species
  int charge;                             // net electrical charge
};
\end{verbatim}

The \verb composition  field in the \verb Species  struct maps the name of
a chemical element to the number of atoms of that type that appear within
the species.

\section{Chemical Reactions}\labelsection{chemical_rxns}

TBD: we currently don't have a candidate for a chemical mechanism, though
we are considering Cantera.

\section{Simulation Context}
\labelsection{context}

To model an aerosol system in Haero, one must first make some decisions about
the aerosol system in question:

\begin{itemize}
  \item What modes are required to characterize the distribution of particle
        sizes?
  \item What species are present in the system, and in what modes are they
        allowed to appear?
  \item What chemical reactions are needed to accurately represent the dynamical
        behavior of the system?
  \item How many atmospheric columns are needed to model the system?
  \item How many vertical levels are needed to resolve the profile of aerosols
        in the system?
  \item Which physical processes are relevant to a simulation of interest?
        Should any processes be excluded, to answer a specific question about
        how the processes interact with each other, or because they are not
        signficant for the system?
\end{itemize}

Clearly, these decisions greatly effect the nature of the model---systems with
different answers to these questions will have very different behavior. When we
define a Haero simulation, we encode the answers to these decisions in a
{\bf context}, represented by a {\verb Context}  object.

What is a context, exactly? One can think about it in a few different ways,
depending on one's perspective:

\begin{itemize}
  \item A context specifies the physics (or the approximations to the physics,
        if you like) represented by an aerosol simulation.
  \item A context stores a set of ``global variables'' that govern the behavior
        for a single aerosol simulation.
  \item Every aerosol simulation gets its own context.
  \item A context describes only the form of an aerosol system's representation.
        It does not describe the state of an aerosol system.
\end{itemize}

Essentially, a context contains all information about a system except for its
(initial) states.

\subsection{The Context Type}

The \verb Context  type that describes a context is opaque, in the sense that
you can query it for information, but this information is accessible only via an
interface consisting of query functions/subroutines. This type is implemented
by a C++ class.

Here's a look at the \verb Context  class and its (public) interface.
\begin{verbatim}
class Context final {
  public:

  // Constructor -- creates a new Context.
  Context(const Parametrizations& parametrizations,
          const std::vector<Mode>& modes,
          const std::vector<Species>& aero_species,
          const std::vector<Species>& gas_species,
          const std::vector<Reaction>& gas_chem_reactions,
          int num_columns, int num_levels);

  // Returns a description of the aerosol parametrizations used by a simulation
  // associated with this context.
  const Parametrizations& parametrizations() const;

  // Returns the list of modes for the context.
  const std::vector<Mode>& modes() const;

  // Returns the list of aerosol species for the context.
  const std::vector<Species>& aero_species() const;

  // Returns the list of gas species for the context.
  const std::vector<Species>& gas_species() const;

  // Returns the number of vertical atmospheric columns in a simulation
  // associated with this context.
  int num_columns() const;

  // Returns the number of vertical levels (cells) in each atmospheric column.
  int num_levels() const;
};
\end{verbatim}

In Fortran, the \verb Context  type consists of an opaque derived type with a
set of associated functions and subroutines. You can find more information
about the Fortran interface in \refappendix{api}.

\section{Simulation State}
\labelsection{state}

When a context has been created that defines an aerosol system for a
simulation, it remains to specify the initial state of that system. The state of
a system is defined in terms of the following prognostic variables:

\begin{itemize}
  \item {\bf modal number densities}: the total number of particles per cubic
        meter in each mode
  \item {\bf aerosol modal mass fractions}: the fraction of a mode occupied
        by each aerosol species present
  \item {\bf gas mole fractions}: the number of moles of each gas species per
        mole of air
\end{itemize}

\section{Parametrizations}
\labelsection{parametrizations}

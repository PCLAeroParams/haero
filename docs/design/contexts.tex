\section{Simulation Contexts}
\labelsection{contexts}

\subsection{Uniform atmosphere}

This is the simplest possible simulation setting, with each level of a physics column containing the exact same environment.  
It provides a means of testing aerosol-specific methods in isolation.

\subsection{Single-column}

In this setting, we use a one-dimensional (vertical) atmosphere and tests of increasing complexity.
The simplest conditions are a  statically stable, hydrostatically balanced atmosphere, with vertical velocity $w=0$.
We assume a linear temperature profile with constant lapse rate $\Gamma = -\partial T/\partial z$,
\begin{equation}
  T(z) = T_0 - \Gamma z,
\end{equation}
where $T_0$ is the reference temperature at $z=0$ (K).  
Note that the atmosphere is statically stable if $\Gamma < \Gamma_{dry}$, where $\Gamma_{dry}$ is the dry adiabatic lapse rate, $\Gamma_{dry} = 0.0098$ K/m and that an isothermal atmosphere is defined by $\Gamma = 0$.
Using the ideal gas law, $p = \rho R T$ and the hydrostatic equation $\partial p/\partial z = -\rho g$, we derive the relations between pressure and height,
\begin{equation}
  p(z) = \begin{cases}
          p_0\exp\left(\frac{-g z}{RT_0}\right) & \Gamma = 0,\\[0.5em]
          p_0\, T_0^{-g/(R\Gamma)}\left(T_0 - \Gamma z\right)^{g/(R\Gamma)} & \Gamma \ne 0,
        \end{cases}
\end{equation}
\begin{equation}
  z(p) = \begin{cases}
         -\frac{R T_0}{g}\log\frac{p}{p_0} & \Gamma = 0,\\[0.5em]
         \frac{T_0}{\Gamma}\left(1 - \left(\frac{p}{p_0}\right)^{R\Gamma/g}\right) & \Gamma \ne 0,
       \end{cases}
\end{equation}
where $p_0$ is the reference pressure at $z=0$ and $T=T_0$.

%A nonzero vertical velocity may be prescribed using a simple analytic function, such as
%\begin{equation}
%  w(z,t) = w_0\cos\left(\frac{2\pi t}{W_P}\right)\sin\left(\frac{2\pi z}{z_{top}}\right),
%\end{equation}
%where $w_0$ is the maximum vertical velocity (m/s), $W_P$ is the velocity period in seconds, and $z_{top}$ is the top of the model (m).
%Note that this velocity has nonzero divergence, $\partial w/\partial z \ne 0$.
%
%\pbc{This velocity probably destroys hydrostatic balance, without similar adjustments to pressure.}

As in \cite{Taylor2020}, we use a pressure-based vertical mass coordinate defined by the arbitrary variable $s$ and the hydrostatic pressure $\pi$ so that $s=0$ at $z=z_{top},~p= p_{top}$ and $s=1$ at $z=0,~p=p_0$.
The column is discretized by $n_{lev}$ vertical levels; pressure is defined at each level, while the height $z$ (or the geopotential $\phi=gz$) is defined at the $n_{lev}+1$ level interfaces (note that both the model top and the surface are interfaces).
Transported quantities such as water vapor and aerosols are defined at level midpoints; the thermodynamic variable ($T$, potential temperature $\theta$, or virtual potential temperature $\theta_v$) are also defined at level midpoints.
Vertical velocity and hydrostatic pressure are defined at interfaces; see \cite{Taylor2020} for details.


\subsubsection{Single column dynamics}

We define dynamics by the vertically Lagrangian form of the HOMME $\theta$-model \cite{Taylor2020}, using the vertical direction only,
\begin{subequations}\label{eq:column_dynamics}
  \begin{align}
    \deriv{w}{t} &= -g(1-\mu),\\
    \deriv{\phi}{t} &= gw, \\
    \deriv{\Theta}{t} &= 0, \\
    \deriv{\partd{\pi}{s}}{t} &= 0,\\
    \deriv{q_v}{t} &= 0,     
  \end{align}
where $\partd{\pi}{s}$ is the pseudo-density, $\mu$ is the ratio of pressure to hydrostatic pressure, $\mu = \partd{p}{s}/\partd{\pi}{s}$. 
In hydrostatic conditions, $\mu \equiv 1$.
The water vapor mixing ratio is $q_v$.
The conservative form of potential temperature $\Theta = \partd{\pi}{s}\theta_v$, where $\theta_v$ is the virtual potential temperature, is related by the equation of state to the geopotential $\phi$,
\begin{equation}
  \partd{\pi}{s} = -R\Theta \frac{\Pi}{p},
\end{equation}
where $\Pi$ is the Exner function,
\begin{equation}
  \Pi =\left(\frac{p}{p_0}\right)^{R/c_p}.%= \left(\frac{R_d}{p_0}\rho\theta_v\right)^{R_d/c_p}
\end{equation}
\end{subequations}
\pbc{Check: 6 equations, 6 unknowns: $w, \phi, \partd{\pi}{s}, q_v, \theta_v, p$.}
For our purposes, it suffices to approximate $\theta_v = \theta(1+0.61q_v)$ as in \cite{KlempWilhelmson1978}, where the potential temperature $\theta = T(p_0/p)^{\kappa}$, with the dry-air constant $\kappa = R_{dry}/c_p$.

Srivastava \cite{Srivastava1967} present a simple model of a one-dimensional cumulus cloud based on warm rain microphysics.
\pbc{Rogers \& Yau provide a better summary; it's 1D Euler plus Kessler microphysics.}
Moisture is represented by $q_v,~q_c,$ and $q_r$, the mixing ratios of water vapor, cloud water droplets, and rain water droplets, respectively.
The model was developed to show the effects of momentum transfer from falling rain on the cloud's vertical velocity; here, we adapt to test various aerosol-related parameterizations.
The column is initially defined by statically unstable hydrostatic balance, with lapse rate $\Gamma = 0.0068$ K/m.
We assume a constant background state with pressure $\overline{p}(z)$, density $\overline{\rho}(z)$ and temperature defined by $\overline{T}(z) = T_0 - \Gamma z$ and hydrostatic balance.
The perturbation temperature $T' = T - \overline{T}(z)$.
The model equations are:
\begin{subequations}
  \begin{align}
    \partd{w}{t} + w\partd{w}{z} & = g\left( \frac{T'}{T} - (q_c + q_r)\right),\\
    \partd{q_v}{t} + w\partd{q_v}{z} &= E_1 + E_2, \\
    \partd{q_c}{t} + w\partd{q_c}{z} &= E_1 - P, \\
    \partd{q_r}{t} + w\partd{q_r}{z} &= - q_r\partd{w_r}{z} - w_r\partd{q_r}{z} - \frac{q_r w_r}{\overline{\rho}}\partdn{\overline{p}}{z}{2}  - E_2 + P,\\
    \partd{T}{t} + w\partd{T}{z} &= -w\Gamma - \frac{L}{c_p}(E_1+E_2),
  \end{align}
\end{subequations}
where $E_1$ describes evaporation and condensation of vapor-to-cloud water, $P$ is the rain water production term that accounts for autoconversion and accretion, $w_r \le 0$ is the rain fall velocity, $L$ is the latent heat of evaporation, and $c_p$ is the specific heat (at constant pressure) of air.

The evaporation rate depends on the saturation of the air,
\begin{equation}
  E_1 = a_0\left(\frac{q_v}{q_{vsat}} - 1\right),
\end{equation}
where $q_{vsat}$ is the saturation mixing ratio given by the Tetens equation,
\begin{equation}
  q_{vsat}(T) = \frac{380.042}{\overline{p}}\exp\left(17.27 \frac{T-273}{T-36}\right).
\end{equation}

%Cloud water in \cite{Srivastava1967} is generated simply,
%\begin{equation*}
%  P_c = -w\partd{q_v}{z};
%\end{equation*}
%we add terms to represent cloud water droplet activation via condensation nuclei,
%\begin{equation}\label{eq:qc_prod}
%  P_c = -w\partd{q_v}{z} + A(T, \Delta T, q_v,q_{aer}),
%\end{equation}
%where $q_{aer}$ is the mixing ratio of aerosol cloud condensation nuclei.
%\todo{Need some help with \eqref{eq:qc_prod}.}
%Rain production is decomposed into three subprocesses, $P_r=P_1 + P_2 + P_3$.
%The first subprocess represents evaporation in convective downdrafts,
%\begin{equation}
%  P_1 = -\mu(w) w \partd{q_v}{z}, \quad \mu(w) = \begin{cases} 1 & w<0, \\ 0 & w >= 0\end{cases}.
%\end{equation}
%The second subprocess represents cloud water conversion,
%\begin{equation}
%  P_2 = \begin{cases} 0 & q_c \le  q_c^{(crit)} \\ \alpha(q_c - q_c^{(crit)}) & q_c >  q_c^{(crit)}\end{cases},
%\end{equation}
%where $\alpha$ is a constant parameter representing the conversion time scale and $q_c^{(crit)}$ is an additional parameter, representing the critical value of cloud water, above which cloud water is converted to rain water.
%\pbc{Perhaps we should consider smoothing out these piecewise functions?}
%The last subprocess represents accretion, the capture of cloud drops by rain drops,
%\pbc{I assume that conversion implies $C: q_c,q_c \mapsto q_r$, whereas accretion implies $Ac:q_c,q_r\mapsto q_r$.}
%The model assumes that rainwater fall speed is proportional to droplet size, $W_r(D) = a D^b$, where $a$ and $b$ are constants, and that rain water droplet sizes are governed by the distribution function $N(D) = N_0e^{-\lambda} D$, where $\lambda = 3.67/D_0$ is a variable parameter related to $D_0$, the median diameter of rainwater droplets \cite[eqn.~(18)]{Srivastava1967}.
%With these assumptions, the expression for accretion is
%\begin{equation}
%  P_3 = \frac{\pi N_0 a \rho \Gamma(3+b) q_r}{4\lambda^{3+b}},
%\end{equation}
%where $\Gamma$ represents the Gamma Function (not a lapse rate).
%The same assumptions give the expression for rain fall velocity,
%\begin{equation}
%  w_r = \frac{a\Gamma(4+b)}{6\lambda^b}.
%\end{equation}

Initial conditions for $w$ and $q_c$ are defined by solving the above equations with $q_r=0$, and $w(z_c)=q_c(z_c) = 0$, $T(z_c) = 277$ K, where $z_c$ is the height of the cloud base.
\todo{Have to figure these out ourselves. Using standard thermodynamics and hydrostatic balance.}
\pbc{I believe cloud base is the first point above $z=0$ where $q_c>0$.}
The bottom boundary condition is $w(0) = 0$.
At $z_{top} = 9.5$ km, if $w(z_top)>0$ then the equations are solved without regard for the boundary; if $w(z_top)<=0$ the other variables are held constant.

%------------------- table: cloud model parameters ---------------
\begin{table}[htbp]
\centering
\caption{Parameters of the 1D cloud model.}
\label{tab:cloud_model_parameters}
\begin{tabular}{ccccc}
  \toprule
  Parameter      &  value 1
                 &  value 2
                 & value 3\\
  \midrule
  $a$ 	 &  1500 cm$^{1/2}$/s  &  --  & -- \\
  $b$    &  0.5                &  --  & --  \\
  $\lambda$  &  1.0            &  --  & --  \\
  $N_0$ &  80 (gm cm)$^{-1}$   &  --  & --  \\
  $\alpha$ & 2.0E-5 s$^{-1}$ (14 hr) & 4.0E-5 s$^{-1}$ (7 hr) & 4.0E-5 s$^{-1}$ (4 min) \\
  $q_c^{(crit)}$ & 0 & 3 gm/kg & 3 gm/kg \\
  \bottomrule
\end{tabular}
\end{table}
%-------------



\subsection{Embedded parameterization}
\chapter{The Haero Driver}
\labelchapter{driver}

The standalone driver program, {\verb haero_driver }, provides a simple way to
explore the capabilities of Haero. It's a simple single-column model with a
bundled one-dimensional dynamics package. With it, you can

\begin{itemize}
  \item run single-column aerosol simulations
  \item perform statistical analysis on ensembles consisting of several columns
  \item conduct time-step convergence studies to build confidence in Haero's
        mathematical algorithms and their implementations
  \item select specific aerosol processes and parametrizations to examine
        in isolation, to debug or verify a given algorithm
  \item study how the aerosol processes interact with one-dimensional dynamics
        and other simplified physical process representations
\end{itemize}

In this chapter we describe the driver and its capabilities. The input format
for the driver is based on YAML and is described in \refappendix{driver_input}.

\section{Column Dynamics}

\subsection{Uniform atmosphere}

This is the simplest possible simulation setting, with each level of a physics column containing the exact same environment.  
It provides a means of testing aerosol-specific methods in isolation.

\subsection{Single-column}

In this setting, we use a one-dimensional (vertical) atmosphere and tests of increasing complexity.
Wherever possible, we choose the same variables, equations, and numerical methods as used by HOMME-NH \cite{Taylor2020}.

\subsubsection{Column dynamics}

We define dynamics by the vertically Lagrangian form of the HOMME $\theta$-model \cite[sec.~2.7]{Taylor2020}, using the vertical direction only,
\begin{subequations}\label{eq:column_dynamics}
  \begin{align}
    \deriv{w}{t} &= -g(1-\mu),\label{eq:vert_vel}\\
    \deriv{\phi}{t} &= gw, \label{eq:geopotential}\\
    \deriv{\Theta}{t} &= 0, \label{eq:theta_v} \\
    \deriv{\partd{\pi}{s}}{t} &= 0, \label{eq:dpids}\\
    \deriv{q_v}{t} &= 0,  \label{eq:q_v}   
  \end{align}
where $\partd{\pi}{s}$ is the pseudo-density, $\mu$ is the ratio of pressure to hydrostatic pressure, $\mu = \partd{p}{s}/\partd{\pi}{s}$. 
In hydrostatic conditions, $\mu \equiv 1$.
The water vapor mixing ratio is $q_v$.
The conservative form of potential temperature $\Theta = \partd{\pi}{s}\theta_v$, where $\theta_v$ is the virtual potential temperature, is related by the equation of state to the geopotential $\phi$,
\begin{equation}
  \partd{\phi}{s} = -R\Theta \frac{\Pi}{p},
\end{equation}
where $\Pi$ is the Exner function,
\begin{equation}
  \Pi =\left(\frac{p}{p_0}\right)^{R/c_p}.%= \left(\frac{R_d}{p_0}\rho\theta_v\right)^{R_d/c_p}
\end{equation}
We have 6 equations for 6 unknowns: $w, \phi, \partd{\pi}{s}, q_v, \theta_v, p$.
\end{subequations}


\begin{rem} 
The zeros on the RHS of equations \eqref{eq:column_dynamics} are due to the Lagrangian 1D setting. 
In the full model, source terms come from the horizontal dynamics, which are excluded here and from flux across levels, which are zero in Lagrangian form.
In the HAERO driver, source terms come from subgrid parameterizations of aerosols and water microphysics via physics/dynamics coupling.  
\end{rem}

These equations are based on a terrain-following vertical coordinate, $s$.
In the context of a single column, the notion of ``terrain'' has no meaning and we may take 
\begin{equation}\label{eq:s_coord}
  s(z) = \frac{z_{top}-z}{z_{top}},
\end{equation}
so that $s=0$ at the model top and increases monotonically to $s=1$ at the surface.
In most settings (including the HOMME $\theta$-model), it is unnecessary to explicitly define $s$; here we do it for convenience only.
The discrete vertical operators from \cite[sec.~4]{Taylor2020} are,
\begin{subequations}
  \begin{align}
    \text{level midpoints: } s_i &= \frac{1}{2}\left(s(z_{i+1/2})+s(z_{i-1/2})\right),\\
    \text{level thickness: } \Delta s_i&= s_{i+1/2}-s_{i-1/2},\quad \Delta s_{i+1/2} = s_{i+1}-s_i,\\
    \text{average to level: } \overline{\phi}_i &=\frac{1}{2}\left(\phi_{i+1/2}+\phi_{i-1/2}\right),\\
    \text{average to interface: } \overline{p}_{i+1/2} &= \frac{(p\Delta s)_{i+1}+(p\Delta s)_i}{2\Delta s_{i+1/2}},\\
    \text{boundary averages: } p_{1/2}&=p_1, ~~p_{n_{lev}+1/2} = p_n,\\
    \text{derivatives at levels: } \left(\partd{\phi}{s}\right)_i &= \frac{\phi_{i+1/2}-\phi_{i-1/2}}{\Delta s_i},\\
    \text{deriavtives at interfaces: } \left(\partd{p}{s}\right)_{i+1/2} &= \frac{p_{i+1}-p_i}{\Delta s_{i+1/2}},\\
    \text{boundary derivatives: } \left(\partd{p}{s}\right)_{1/2} &= \frac{p_1-p_{1/2}}{\frac{1}{2}\Delta s_{1/2}},~~\left(\partd{p}{s}\right)_{n_{lev}+1/2} = \frac{p_{n_{lev}+1/2}-p_n}{\frac{1}{2}\Delta s_{n_{lev}}},
  \end{align}
  in our case, \eqref{eq:s_coord} is used in each one to compute $s$ and $\Delta s$.
\end{subequations}


\subsubsection{Initialization}

To conform with the HOMME-NH dynamical core \cite{Taylor2020}, we use  virtual temperature $T_v$, instead of temperature $T$.
Virtual temperature may be approximated by \cite[eqn.~(2.1)]{KlempWilhelmson1978},
\begin{equation}\label{eq:virtual_temperature}
  T_v(z) = T(z)(1+\alpha_q q_v(z)),
\end{equation}
where $q_v$ is the water vapor mass mixing ratio and the constant $\alpha_q = 0.61$ K.

The initial water vapor profile can be characterized by exponential decay with height \cite{Hashimoto2005,Palchetti2008}; here it is defined by two parameters, $q_v^{(0)}$ and $q_v^{(1)}$, the mixing ratio value at $z=0$ and its decay rate with height, respectively;
\begin{equation}\label{eq:qv_profile}
  q_v(z) = q_v^{(0)}e^{-q_v^{(1)}z}.
\end{equation}

For simplicity we have assumed the equation of state implied by \eqref{eq:virtual_temperature}, from \cite{KlempWilhelmson1978}, which allows us to use the gas constant for dry air.  
   HOMME-NH uses a slightly different approximation; see \cite[sec.~2.3]{Taylor2020}.
   
We assume a linear virtual temperature profile with constant lapse rate $\Gamma = -\partial T_v/\partial z$,
\begin{equation}\label{eq:temperature_profile}
  T_v(z) = T_0 - \Gamma z,
\end{equation}
where $T_0$ is the reference virtual temperature at $z=0$.  

\begin{rem}
  Due to the formulation of \eqref{eq:temperature_profile} in virtual temperature, standard notions of static stability via comparison of the environmental lapse rate with the dry adiabatic lapse rate are not applicable; conversion of virtual temperature to temperature using \eqref{eq:virtual_temperature} is required for this analysis.
\end{rem}

Using the ideal gas law for moist air, $p = \rho R_d T_v$ and the hydrostatic equation $\partial p/\partial z = -\rho g$, we derive the relations between pressure and height,
\begin{equation}\label{eq:hydrostatic_pressure_profile}
  p(z) = \begin{cases}
          p_0\exp\left(\frac{-g z}{R_dT_0}\right) & \Gamma = 0,\\[0.5em]
          p_0\, T_0^{-g/(R\Gamma)}\left(T_0 - \Gamma z\right)^{g/(R_d\Gamma)} & \Gamma \ne 0,
        \end{cases}
\end{equation}
\begin{equation}\label{eq:hydrostatic_height_profile}
  z(p) = \begin{cases}
         -\frac{R T_0}{g}\log\frac{p}{p_0} & \Gamma = 0,\\[0.5em]
         \frac{T_0}{\Gamma}\left(1 - \left(\frac{p}{p_0}\right)^{R\Gamma/g}\right) & \Gamma \ne 0,
       \end{cases}
\end{equation}
where $p_0$ is the reference pressure at $z=0$ and $T=T_0$.

Column data are initialized as follows:
\begin{enumerate}
  \item A set of levels and interfaces are defined in one of two ways:
  \begin{enumerate}
    \item Height level interfaces $z_{1/2}, z_{3/2}, \dotsc, z_{n_{lev}+1/2}$  are chosen such that $z_{1/2} = z_{top}$ and $z_{n_{lev}+1/2} = 0$.
        Geopotential $\phi(z_{i+1/2}) = g z_{i+1/2}$ is defined for $i=0,\dotsc,n_{lev}$.
    \begin{itemize}
      \item Level midpoints are initialized so that $z_i = (z_{i+1/2} + z_{i-1/2})/2$ for $i=1,\dotsc,n_{lev}$. 
      \item Level and interface values for $s$ and $\Delta s$ are computed using \eqref{eq:s_coord}.
      \item Given a user-specified lapse rate $\Gamma$, hydrostatic pressure $\pi$ is initialized at each level interface and pressure $p$ is initialized at level midpoints using \eqref{eq:hydrostatic_pressure_profile}.
    \end{itemize}
    \item Pressure level interfaces $\pi_{1/2}, \pi_{3/2}, \dotsc, \pi_{n_{lev}+1/2}$  are chosen such that $\pi_{1/2} = \pi_{top}$ and $\pi_{n_{lev}+1/2} = p_0$.
    \begin{itemize}
      \item Given a user-specified lapse rate $\Gamma$, geopotential $\phi=gz$ is initialized at each level interface using \eqref{eq:hydrostatic_height_profile}.
      \item Level midpoints are initialized so that $z_i = (z_{i+1/2} + z_{i-1/2})/2$ for $i=1,\dotsc,n_{lev}$, and level \& interface values for $s$ and $\Delta s$ are defined using \eqref{eq:s_coord}.
      \item Pressure $p$ is defined via \eqref{eq:hydrostatic_pressure_profile} at level midpoints. 
    \end{itemize}
  \end{enumerate}
  \item Pseudodensity is defined at each level midpoint,
    \begin{equation}
      \left(\partd{\pi}{s}\right)_i = -\frac{g(\pi_{i+1/2}-\pi_{i-1/2})}{z_{top}(\phi_{i+1/2}-\phi_{i-1/2})},
    \end{equation}
    where we have used $\partial s / \partial z = - 1 /z_{top}$ from \eqref{eq:s_coord}.
%    \begin{rem}
%    This is an example of the $s$-coordinate cancelling, which is why most models don't explicitly define it.
%    \end{rem}
  \item A water vapor profile is defined, by choosing constant values for $q_v^{(0)}$ and $q_v^{(1)}$ in \eqref{eq:qv_profile}, and its values are stored at level midpoints.
  \item Virtual potential temperature $\theta_v = T_v(p_0/p)^{\kappa}$, with the dry-air constant $\kappa = R_{dry}/c_p$, is defined at level midpoints.
  \item The initial velocity profile is defined, usually $w=0$.  
\end{enumerate}

\subsubsection{Time stepping}

%We use an explicit 2nd order Runge-Kutta method, hereafter RK2, for advancing \eqref{eq:column_dynamics} forward in time \cite[\S 8.3.3]{DahlquistBjork}.

\paragraph{Case 1: Prescribed velocity.} In this case we assume that vertical velocity $w$ is prescribed by a known function $w=w(z,t)$, for example,
\begin{equation}\label{eq:presc_velocity}
  w(z,t) = w_0 \sin \frac{2\pi t}{t_p}\sin \frac{m \pi z}{z_{top}},
\end{equation}
where the user-chosen parameters are $w_0$, the maximum vertical velocity [m/s], $t_p$, the period [s], and $m$, the vertical wave number (a dimensionless integer).
%Equations \eqref{eq:column_dynamics} are separable in this case and we can solve them analytically by recalling that $z = \phi/g$.
Using \eqref{eq:presc_velocity} in \eqref{eq:column_dynamics} and recalling that $\phi = gz$, the geopotential equation becomes,
\begin{equation}
  \deriv{\phi}{t} = gw_0\sin\frac{2\pi t}{t_p}\sin\frac{m\pi \phi(t)}{g z_{top}},
\end{equation}
which is a separable equation for $\phi(t)$ that can be solved analytically for each interface.
\begin{subequations}
We find that
\begin{equation}
  \phi_{i+1/2}(t_{n+1}) = \frac{2 g z_{top}}{m \pi}\arccot\left[ {\exp\left(-\frac{m t_p w_0}{z_{top}}\sin^2\frac{\pi t_{n+1}}{t_p}\right)}{{\cot\left(\frac{m \pi \phi_0}{2gz_{top}}\right)}}\right],
\end{equation}
where $\phi_0 = \phi_{i+1/2}(0)$.
Vertical velocity is updated as
\begin{equation}
  w_{i+1/2}(t_{n+1}) = w(\phi_{i+1/2}(t_{n+1})/g, t_{n+1}).
\end{equation}

The level variables lack source/sink terms; hence,
\begin{align}
  \partd{\pi}{s}_i(t_{n+1}) &= \partd{\pi}{s}_i(t_n),\\
  \theta_{v,i}(t_{n+1}) &= \theta_{v,i}(t_{n+1}),\\
  q_{v,i}(t_{n+1}) &= q_{v,i}(t_{n+1}).
\end{align}

Substituting \eqref{eq:presc_velocity} into the momentum equation we find that
\begin{equation}
  \mu_{i+1/2}(t_{n+1})= 1 + \frac{2\pi w_0}{gt_p}\cos\frac{2\pi t_{n+1}}{t_p}\sin\frac{m\pi \phi_{i+1/2}(t_{n+1})}{gz_{top}}.
\end{equation}

At each level we can directly apply the vertical derivative operator to get
\begin{equation}
  \partd{\phi}{s}_i(t_{n+1}) = \frac{\phi_{i+1/2}(t_{n+1})-\phi_{i-1/2}(t_{n+1})}{\Delta s_i}.
\end{equation}
After averaging the pseudodensity to the interfaces, the definition of $\mu$ gives $\partd{p}{s}$,
\begin{equation}
  \partd{p}{s}_{i+1/2}(t_{n+1}) = \mu_{i+1/2}(t_{n+1})\overline{\partd{\pi}{s}}_{i+1/2}(t_{n+1}).
\end{equation}
Pressure at the levels follows from the equation of state,
\begin{equation}
  p_{i+1}(t_{n+1}) = \exp\left(\frac{1}{{R}/{c_p}-1}\right)\left(\frac{\partd{\phi}{s}_ip_0^{R/c_p}}{R \theta_{v,i}(t_{n+1}) \partd{\pi}{s}_i(t_{n+1})}\right).
\end{equation}
Finally, surface pressure is updated as \cite[eqn.~(35)]{Taylor2020},
\begin{equation}
  p_s(t_{n+1}) = p_{top} -\frac{1}{2}\left[\left(\partd{p}{s}\Delta s\right)_{1/2} + \left(\partd{p}{s}\Delta s\right)_{n_{lev}+1/2}\right] + \sum_{k=0}^{n_{lev}}\left(\partd{p}{s}\Delta s\right)_{k+1/2}.
\end{equation}
\end{subequations}



\subsubsection{Simple microphysics}

A simple cloud model with warm-rain microphysics, often called \emph{Kessler microphysics}, is summarized in \cite[ch.~15]{RogersYau}.
It introduces mass mixing ratio tracers for cloud liquid water $q_c$ and rain water $q_r$ and is very similar to the microphysics used in \cite{SoongOgura1973,KlempWilhelmson1978}, and source terms for the dynamics equations.

\paragraph{Vertical velocity.} The vertical velocity $w$ is adjusted to account for falling liquid water, so that \eqref{eq:vert_vel} is now
\begin{equation*}
  \deriv{w}{t} = -g(1-\mu -(q_c+q_r)).  
\end{equation*}


\paragraph{Moisture variables.}
We assume, following \cite{SoongOgura1973,KlempWilhelmson1978}, that any supersaturated immediately condenses, and that any cloud liquid present in unsaturated air immediately evaporates.
Rain water only evaporates if $q_c = 0$.
These processes are represented by terms $E_1:q_v\leftrightarrow q_c$ and $E_2:q_v \leftarrow q_r$.
Equations \eqref{eq:theta_v} and \eqref{eq:q_v} become,
\begin{align}
  \deriv{\Theta}{t} &= \frac{L}{c_p \Pi }(E_1 + E_2), \\
  \deriv{q_v}{t} &= E_1 + E_2,
\end{align}
and new equations are added for $q_c$ and $q_r$:
\begin{align}
  \deriv{q_c}{t} &= -E_1 - (P_1 + P_2),\label{eq:cloud_water} \\
  \deriv{q_r}{t} &= -\frac{1}{\rho}\partd{}{z}(\rho q_r w_r) - E_2 + (P_1 + P_2), \label{eq:rain_water}
\end{align}
where $P_1:q_c+q_c\mapsto q_r$ and $P_2:q_c+q_r\mapsto q_r$ represent rain production from autoconversion and accretion, respectively, and $w_r$ is the velocity of rain water. Each is discussed below in greater detail.

Temperature $T$ and potential temperature $\theta$ are recovered from $\theta_v$ at level midpoints as
\begin{equation}\label{eq:temperature}
  T = \frac{\theta_v}{1+\alpha_v q_v} \left(\frac{p_0}{p}\right)^{-\kappa}, \quad \theta = \frac{\theta_v}{1+\alpha_vq_v},
\end{equation}
and the saturation mixing ratio is given by the Tetens equation,
\begin{equation}\label{eq:tetens}
  q_{vs}(T) = \frac{380.042}{p}\exp\left(\frac{15}{2}\log(10) \frac{T-273}{T-36}\right).
\end{equation}

  
\paragraph{Evaporation and condensation.}
The evaporation/condensation rate is parameterized by \cite[eqn.~(5)]{Srivastava1967}
\begin{equation}\label{eq:evap_param}
  E_1 = w\partd{q_v}{z}.
\end{equation}
Rain water is only allowed to evaporate in downdrafts and external to the cloud  \cite[eqn.~(10)]{Srivastava1967},
\begin{equation}\label{eq:rain_evap_param}
  E_2 = -\alpha_E w \partd{q_v}{z}, \quad \alpha_E = \begin{cases} 1 & w<0 \text{ and } q_c = 0,\\
 0 & \text{otherwise}  
 \end{cases}.
\end{equation}

\begin{rem}
Parameterizations \eqref{eq:evap_param} and \eqref{eq:rain_evap_param} are not meant to be interpreted as useful for physically realistic cloud simulations. 
They simply provide a basic model which is useful for testing purposes.
See \cite{SoongOgura1973,KlempWilhelmson1978} for more advanced parameterizations based on the difference between $q_v$ and the saturation mixing ratio, $q_{vs}$.
\end{rem}

\paragraph{Autoconversion.} Autoconversion only occurs in the presence of sufficient cloud water droplets.
Here, ``sufficient'' is defined as greater than a constant critical value, $q_c^{(crit)}$, and \cite[eqn.~(12)]{Srivastava1967}
\begin{equation}
  P_1 = \begin{cases}
    0 & q_c \le q_c^{(crit)}, \\
    \alpha_{auto}(q_c - q_c^{(crit)}) & q_c > q_c^{(crit)},
  \end{cases}
\end{equation}
where $\alpha_{auto}$ [1/s] is the inverse of the autoconversion time scale and $q_c^{(crit)}$ is a user-defined parameter.

\paragraph{Accretion.} Accretion describes the capture of cloud water droplets by rainwater droplets.
As in \cite{SoongOgura1973,KlempWilhelmson1978}, we use
\begin{equation}
  P_2 = \alpha_{accr}q_cq_r^{7/8},
\end{equation}
with $\alpha_{accr} = 2.2$.

\paragraph{Rainwater flux.} Rainwater by definition falls relative to the air; its vertical flux is described by the first term on the RHS of \eqref{eq:rain_water}.
  We use \cite[eqns.~(14),(15)]{SoongOgura1973} to approximate $w_r$,
  \begin{equation}
    w_r = 36.34 (\rho q_r)^{0.1364} ~\text{m/s}.
  \end{equation}
  

\subsection{Embedded parameterization}

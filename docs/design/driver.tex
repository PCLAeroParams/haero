\chapter{The Haero Driver}
\labelchapter{driver}

The standalone driver program, {\verb haero_driver }, provides a simple way to
explore the capabilities of Haero. It's a simple single-column model with a
bundled one-dimensional dynamics package. With it, you can

\begin{itemize}
  \item run single-column aerosol simulations
  \item perform statistical analysis on ensembles consisting of several columns
  \item conduct time-step convergence studies to build confidence in Haero's
        mathematical algorithms and their implementations
  \item select specific aerosol processes and parametrizations to examine
        in isolation, to debug or verify a given algorithm
  \item study how the aerosol processes interact with one-dimensional dynamics
        and other simplified physical process representations
\end{itemize}

In this chapter we describe the driver and its capabilities. The input format
for the driver is based on YAML and is described in \refappendix{driver_input}.

\section{Column Dynamics}

\subsection{Uniform atmosphere}

This is the simplest possible simulation setting, with each level of a physics column containing the exact same environment.  
It provides a means of testing aerosol-specific methods in isolation.

\subsection{Single-column}

In this setting, we use a one-dimensional (vertical) atmosphere and tests of increasing complexity.
Wherever possible, we choose the same variables, equations, and numerical methods as used by HOMME-NH \cite{Taylor2020}.

\subsubsection{Column dynamics}

We define dynamics by the vertically Lagrangian form of the HOMME $\theta$-model \cite[sec.~2.7]{Taylor2020}, using the vertical direction only,
\begin{subequations}\label{eq:column_dynamics}
  \begin{align}
    \deriv{w}{t} &= -g(1-\mu),\label{eq:vert_vel}\\
    \deriv{\phi}{t} &= gw, \label{eq:geopotential}\\
    \deriv{\Theta}{t} &= 0, \label{eq:theta_v} \\
    \deriv{\partd{\pi}{s}}{t} &= 0, \label{eq:dpids}\\
    \deriv{q_v}{t} &= 0,  \label{eq:q_v}   
  \end{align}
where $\partd{\pi}{s}$ is the pseudo-density, $\mu$ is the ratio of pressure to hydrostatic pressure, $\mu = \partd{p}{s}/\partd{\pi}{s}$. 
In hydrostatic conditions, $\mu \equiv 1$.
The water vapor mixing ratio is $q_v$.
The conservative form of potential temperature $\Theta = \partd{\pi}{s}\theta_v$, where $\theta_v$ is the virtual potential temperature, is related by the equation of state to the geopotential $\phi$,
\begin{equation}
  \partd{\pi}{s} = -R\Theta \frac{\Pi}{p},
\end{equation}
where $\Pi$ is the Exner function,
\begin{equation}
  \Pi =\left(\frac{p}{p_0}\right)^{R/c_p}.%= \left(\frac{R_d}{p_0}\rho\theta_v\right)^{R_d/c_p}
\end{equation}
We have 6 equations for 6 unknowns: $w, \phi, \partd{\pi}{s}, q_v, \theta_v, p$.
\end{subequations}


\begin{rem} 
The zeros on the RHS of equations \eqref{eq:column_dynamics} are due to the Lagrangian 1D setting. 
In the full model, source terms come from the horizontal dynamics, which are excluded here and from flux across levels, which are zero in Lagrangian form.
In the HAERO driver, source terms come from subgrid parameterizations of aerosols and water microphysics via physics/dynamics coupling.  
\end{rem}

These equations are based on a terrain-following vertical coordinate, $s$.
In the context of a single column, the notion of ``terrain'' has no meaning and we may take 
\begin{equation}\label{eq:s_coord}
  s(z) = \frac{z_{top}-z}{z_{top}},
\end{equation}
so that $s=0$ at the model top and increases monotonically to $s=1$ at the surface.
In most settings (include the HOMME $\theta$-model), it is unnecessary to explicitly define $s$; here we do it for convenience only.
The discrete vertical operators from \cite[sec.~4]{Taylor2020} are,
\begin{subequations}
  \begin{align}
    \text{level midpoints: } s_i &= \frac{1}{2}\left(s(z_{i+1/2})+s(z_{i-1/2})\right),\\
    \text{level thickness: } \Delta s_i&= s_{i+1/2}-s_{i-1/2},\quad \Delta s_{i+1/2} = s_{i+1}-s_i,\\
    \text{average to level: } \overline{\phi}_i &=\frac{1}{2}\left(\phi_{i+1/2}+\phi_{i-1/2}\right),\\
    \text{average to interface: } \overline{p}_{i+1/2} &= \frac{(p\Delta s)_{i+1}+(p\Delta s)_i}{2\Delta s_{i+1/2}},\\
    \text{boundary averages: } p_{1/2}&=p_1, ~~p_{n_{lev}+1/2} = p_n,\\
    \text{derivatives at levels: } \left(\partd{\phi}{s}\right)_i &= \frac{\phi_{i+1/2}-\phi_{i-1/2}}{\Delta s_i},\\
    \text{deriavtives at interfaces: } \left(\partd{p}{s}\right)_{i+1/2} &= \frac{p_{i+1}-p_i}{\Delta s_{i+1/2}},\\
    \text{boundary derivatives: } \left(\partd{p}{s}\right)_{1/2} &= \frac{p_1-p_{1/2}}{\frac{1}{2}\Delta s_{1/2}},~~\left(\partd{p}{s}\right)_{n_{lev}+1/2} = \frac{p_{n_{lev}+1/2}-p_n}{\frac{1}{2}\Delta s_{n_{lev}}},
  \end{align}
  in our case, \eqref{eq:s_coord} is used in each one to compute $s$ and $\Delta s$.
\end{subequations}


\subsubsection{Initialization}

The initial water vapor profile can be characterized by exponential decay with height \cite{Hashimoto2005,Palchetti2008}; here it is defined by two parameters, $q_v^{(0)}$ and $q_v^{(1)}$, the mixing ratio value at $z=0$ and its decay rate with height, respectively;
\begin{equation}\label{eq:qv_profile}
  q_v(z) = q_v^{(0)}e^{-q_v^{(1)}z}.
\end{equation}
To conform with the HOMME-NH dynamical core \cite{Taylor2020}, we use  virtual temperature $T_v$, which may be approximated by \cite[eqn.~(2.1)]{KlempWilhelmson1978},
\begin{equation}\label{eq:virtual_temperature}
  T_v(z) = T(z)(1+\alpha_q q_v(z)),
\end{equation}
where $q_v$ is the water vapor mass mixing ratio and the constant $\alpha_q = 0.61$ K.
For simplicity we have assumed the equation of state implied by \eqref{eq:virtual_temperature}, from \cite{KlempWilhelmson1978}, which allows us to use the gas constant for dry air.  
   HOMME-NH uses a slightly different approximation; see \cite[sec.~2.3]{Taylor2020}.
   
We assume a linear virtual temperature profile with constant lapse rate $\Gamma = -\partial T_v/\partial z$,
\begin{equation}\label{eq:temperature_profile}
  T_v(z) = T_0 - \Gamma z,
\end{equation}
where $T_0$ is the reference virtual temperature at $z=0$.  

\begin{rem}
  Due to the formulation of \eqref{eq:temperature_profile} in virtual temperature, standard notions of static stability via comparison of the environmental lapse rate with the dry adiabatic lapse rate are not applicable; conversion of virtual temperature to temperature using \eqref{eq:virtual_temperature} is required for this analysis.
\end{rem}

Using the ideal gas law for moist air, $p = \rho R_d T_v$ and the hydrostatic equation $\partial p/\partial z = -\rho g$, we derive the relations between pressure and height,
\begin{equation}\label{eq:hydrostatic_pressure_profile}
  p(z) = \begin{cases}
          p_0\exp\left(\frac{-g z}{R_dT_0}\right) & \Gamma = 0,\\[0.5em]
          p_0\, T_0^{-g/(R\Gamma)}\left(T_0 - \Gamma z\right)^{g/(R_d\Gamma)} & \Gamma \ne 0,
        \end{cases}
\end{equation}
\begin{equation}\label{eq:hydrostatic_height_profile}
  z(p) = \begin{cases}
         -\frac{R T_0}{g}\log\frac{p}{p_0} & \Gamma = 0,\\[0.5em]
         \frac{T_0}{\Gamma}\left(1 - \left(\frac{p}{p_0}\right)^{R\Gamma/g}\right) & \Gamma \ne 0,
       \end{cases}
\end{equation}
where $p_0$ is the reference pressure at $z=0$ and $T=T_0$.

Column data are initialized as follows:
\begin{enumerate}
  \item A set of levels and interfaces are defined in one of two ways:
  \begin{enumerate}
    \item Height level interfaces $z_{1/2}, z_{3/2}, \dotsc, z_{n_{lev}+1/2}$  are chosen such that $z_{1/2} = z_{top}$ and $z_{n_{lev}+1/2} = 0$.
        Geopotential $\phi(z_{i+1/2}) = g z_{i+1/2}$ is defined for $i=0,\dotsc,n_{lev}$.
    \begin{itemize}
      \item Level midpoints are initialized so that $z_i = (z_{i+1/2} + z_{i-1/2})/2$ for $i=1,\dotsc,n_{lev}$. 
      \item Level and interface values for $s$ and $\Delta s$ are computed using \eqref{eq:s_coord}.
      \item Given a user-specified lapse rate $\Gamma$, hydrostatic pressure $\pi$ is initialized at each level interface and pressure $p$ is initialized at level midpoints using \eqref{eq:hydrostatic_pressure_profile}.
    \end{itemize}
    \item Pressure level interfaces $\pi_{1/2}, \pi_{3/2}, \dotsc, \pi_{n_{lev}+1/2}$  are chosen such that $\pi_{1/2} = \pi_{top}$ and $\pi_{n_{lev}+1/2} = p_0$.
    \begin{itemize}
      \item Given a user-specified lapse rate $\Gamma$, geopotential $\phi=gz$ is initialized at each level interface using \eqref{eq:hydrostatic_height_profile}.
      \item Level midpoints are initialized so that $z_i = (z_{i+1/2} + z_{i-1/2})/2$ for $i=1,\dotsc,n_{lev}$, and level \& interface values for $s$ and $\Delta s$ are defined using \eqref{eq:s_coord}.
      \item Pressure $p$ is defined via \eqref{eq:hydrostatic_pressure_profile} at level midpoints. 
    \end{itemize}
  \end{enumerate}
  \item Pseudodensity is defined at each level midpoint,
    \begin{equation}
      \left(\partd{\pi}{s}\right)_i = -\frac{g(\pi_{i+1/2}-\pi_{i-1/2})}{z_{top}(\phi_{i+1/2}-\phi_{i-1/2})},
    \end{equation}
    where we have used $\partial s / \partial z = - 1 /z_{top}$ from \eqref{eq:s_coord}.
    \begin{rem}
    This is an example of the $s$-coordinate cancelling, which is why most models don't explicitly define it.
    \end{rem}
  \item A water vapor profile is defined, by choosing constant values for $q_v^{(0)}$ and $q_v^{(1)}$ in \eqref{eq:qv_profile}, and its values are stored at level midpoints.
  \item Virtual potential temperature $\theta_v = T_v(p_0/p)^{\kappa}$, with the dry-air constant $\kappa = R_{dry}/c_p$, is defined at level midpoints.
  \item The initial velocity profile is defined, usually $w=0$.  
\end{enumerate}


\subsubsection{Simple microphysics}

A simple cloud model with warm-rain microphysics, often called \emph{Kessler microphysics}, is summarized in \cite[ch.~15]{RogersYau}.
It introduces mass mixing ratio tracers for cloud liquid water $q_c$ and rain water $q_r$ and is very similar to the microphysics used in \cite{SoongOgura1973,KlempWilhelmson1978}, and source terms for the dynamics equations.

\paragraph{Vertical velocity.} The vertical velocity $w$ is adjusted to account for falling liquid water, so that \eqref{eq:vert_vel} is now
\begin{equation*}
  \deriv{w}{t} = -g(1-\mu -(q_c+q_r)).  
\end{equation*}


\paragraph{Moisture variables.}
We assume, following \cite{SoongOgura1973,KlempWilhelmson1978}, that any supersaturated immediately condenses, and that any cloud liquid present in unsaturated air immediately evaporates.
Rain water only evaporates if $q_c = 0$.
These processes are represented by terms $E_1:q_v\leftrightarrow q_c$ and $E_2:q_v \leftarrow q_r$.
Equations \eqref{eq:theta_v} and \eqref{eq:q_v} become,
\begin{align}
  \deriv{\Theta}{t} &= \frac{L}{c_p \Pi }(E_1 + E_2), \\
  \deriv{q_v}{t} &= E_1 + E_2,
\end{align}
and new equations are added for $q_c$ and $q_r$:
\begin{align}
  \deriv{q_c}{t} &= -E_1 - (P_1 + P_2),\\
  \deriv{q_r}{t} &= -\frac{1}{\rho}\partd{}{z}(\rho q_r w_r) - E_2 + (P_1 + P_2),
\end{align}
where $P_1:q_c+q_c\mapsto q_r$ and $P_2:q_c+q_r\mapsto q_r$ represent rain production from autoconversion and accretion, respectively, and $w_r$ is the velocity of rain water. Each is discussed below in greater detail.

Temperature $T$ and potential temperature $\theta$ are recovered from $\theta_v$ at level midpoints as
\begin{equation}\label{eq:temperature}
  T = \frac{\theta_v}{1+\alpha_v q_v} \left(\frac{p_0}{p}\right)^{-\kappa}, \quad \theta = \frac{\theta_v}{1+\alpha_vq_v},
\end{equation}
and the saturation mixing ratio is given by the Tetens equation,
\begin{equation}\label{eq:tetens}
  q_{vs}(T) = \frac{380.042}{p}\exp\left(\frac{15}{2}\log(10) \frac{T-273}{T-36}\right).
\end{equation}

% Evaporation rate is modeled as a linear function of the difference between the vapor mixing ratio $q_v$ and the saturation mixing ratio $q_{vs}$.
%Temperature $T$ and potential temperature are recovered from $\theta_v$ at level midpoints as
%\begin{equation}\label{eq:temperature}
%  T = \frac{\theta_v}{1+\alpha_v q_v} \left(\frac{p_0}{p}\right)^{-\kappa}, \quad \theta = \frac{\theta_v}{1+\alpha_vq_v},
%\end{equation}
%and the saturation mixing ratio is given by the Tetens equation,
%\begin{equation}
%  q_{vs}(T) = \frac{380.042}{p}\exp\left(\frac{15}{2}\log(10) \frac{T-273}{T-36}\right).
%\end{equation}
\paragraph{Evaporation and condensation.}
The evaporation/condensation rate is parameterized by \cite[eqn.~(5)]{Srivastava1967}
\begin{equation}\label{eq:evap_param}
  E_1 = w\partd{q_v}{z}.
\end{equation}
Rain water is only allowed to evaporate in downdrafts and external to the cloud  \cite[eqn.~(10)]{Srivastava1967},
\begin{equation}\label{eq:rain_evap_param}
  E_2 = -\alpha_E w \partd{q_v}{z}, \quad \alpha_E = \begin{cases} 1 & w<0 \text{ and } q_c = 0,\\
 0 & \text{otherwise}  
 \end{cases}.
\end{equation}

\begin{rem}
Parameterizations \eqref{eq:evap_param} and \eqref{eq:rain_evap_param} are not meant to be interpreted as useful for physically realistic cloud simulations. 
They simply provide a basic model which is useful for testing purposes.
See \cite{SoongOgura1973,KlempWilhelmson1978} for more advanced parameterizations based on the difference between $q_v$ and the saturation mixing ratio, $q_{vs}$.
\end{rem}

\paragraph{Autoconversion.} Autoconversion only occurs in the presence of sufficient cloud water droplets.
Here, ``sufficient'' is defined as greater than a constant critical value, $q_c^{(crit)}$, and \cite[eqn.~(12)]{Srivastava1967}
\begin{equation}
  P_1 = \begin{cases}
    0 & q_c \le q_c^{(crit)}, \\
    \alpha_{auto}(q_c - q_c^{(crit)}) & q_c > q_c^{(crit)},
  \end{cases}
\end{equation}
where $\alpha_{auto}$ [1/s] is the inverse of the autoconversion time scale and $q_c^{(crit)}$ is a user-defined parameter.


% First the equations are evolved assuming that $E_1 = 0$; denote these values $\theta_v^*$, $p^*$, $q_v^*$, and $q_c^*$.
%Then, adjustments are made as follows.
%
%We denote by $T^*$ the value of the unadjusted state's temperature, \eqref{eq:temperature} at $\theta_v=\theta_v^*$; similarly, $q_{vs}^* = q_{vs}(T^*)$ is the saturation mixing ratio corresponding to the unadjusted state.
%
%\begin{itemize}
%  \item Case 1: Saturated air, $q_v=q_{vs}$.  
%    This is an equilibriums state where the evaporation and condensation rates are equal and opposite, $E_1 = E_2 = 0$.
%    Hence, $\theta_v = \theta_v^*$, $p=p^*$, $q_v=q_v^*$ and $q_c=q_c^*$.
%  \item Case 2: Supersaturated air, $q_v > q_{vs}$; water vapor condenses into cloud water and $E_2 = 0$.  
%      We follow Klemp \& Wilhelmson's \cite[eqns.~(3.8)-(3.9)]{KlempWilhelmson1978} summary of \cite{SoongOgura1973}.
%      We define the quantity,
%      \begin{equation}
%        \beta_E = q_{vs}^*\left(1 + \frac{4093 \alpha_E\Pi}{(T^*-36)^2}\right).
%      \end{equation} 
%      Then we compute the updated values,
%      \begin{equation}
%        \theta^{n+1} = \theta^* + \frac{L}{c_p\Pi}\left(1 + \frac{L}{c_p\Pi}\beta_E\right)^{-1}q_v^*,
%      \end{equation}
%      \begin{equation}
%        q_v^{n+1} = \frac{c_p\Pi}{L}(\theta^*-\theta^{n+1}) + q_v^*.
%      \end{equation}
%      
%      
%  \item Case 3: Unsaturated air, $q_v < q_{vs}$; evaporation.
%    \begin{itemize}
%      \item Case 3a: $q_c>0$.
%      \item Case 3b: $q_c=0$.
%    \end{itemize}
%\end{itemize}



%For our purposes, it suffices to approximate $\theta_v = \theta(1+0.61q_v)$ as in \cite[eqn.~(2.1)]{KlempWilhelmson1978}, where the potential temperature $\theta = T(p_0/p)^{\kappa}$, with the dry-air constant $\kappa = R_{dry}/c_p$.
%
%Srivastava \cite{Srivastava1967} present a simple model of a one-dimensional cumulus cloud based on warm rain microphysics.
%\pbc{Rogers \& Yau provide a better summary; it's 1D Euler plus Kessler microphysics.}
%Moisture is represented by $q_v,~q_c,$ and $q_r$, the mixing ratios of water vapor, cloud water droplets, and rain water droplets, respectively.
%The model was developed to show the effects of momentum transfer from falling rain on the cloud's vertical velocity; here, we adapt to test various aerosol-related parameterizations.
%The column is initially defined by statically unstable hydrostatic balance, with lapse rate $\Gamma = 0.0068$ K/m.
%We assume a constant background state with pressure $\overline{p}(z)$, density $\overline{\rho}(z)$ and temperature defined by $\overline{T}(z) = T_0 - \Gamma z$ and hydrostatic balance.
%The perturbation temperature $T' = T - \overline{T}(z)$.
%The model equations are:
%\begin{subequations}
%  \begin{align}
%    \partd{w}{t} + w\partd{w}{z} & = g\left( \frac{T'}{T} - (q_c + q_r)\right),\\
%    \partd{q_v}{t} + w\partd{q_v}{z} &= E_1 + E_2, \\
%    \partd{q_c}{t} + w\partd{q_c}{z} &= E_1 - P, \\
%    \partd{q_r}{t} + w\partd{q_r}{z} &= - q_r\partd{w_r}{z} - w_r\partd{q_r}{z} - \frac{q_r w_r}{\overline{\rho}}\partdn{\overline{p}}{z}{2}  - E_2 + P,\\
%    \partd{T}{t} + w\partd{T}{z} &= -w\Gamma - \frac{L}{c_p}(E_1+E_2),
%  \end{align}
%\end{subequations}
%where $E_1$ describes evaporation and condensation of vapor-to-cloud water, $P$ is the rain water production term that accounts for autoconversion and accretion, $w_r \le 0$ is the rain fall velocity, $L$ is the latent heat of evaporation, and $c_p$ is the specific heat (at constant pressure) of air.
%
%The evaporation rate depends on the saturation of the air,
%\begin{equation}
%  E_1 = a_0\left(\frac{q_v}{q_{vsat}} - 1\right),
%\end{equation}
%where $q_{vsat}$ is the saturation mixing ratio given by the Tetens equation,
%\begin{equation}
%  q_{vsat}(T) = \frac{380.042}{\overline{p}}\exp\left(17.27 \frac{T-273}{T-36}\right).
%\end{equation}
%
%Cloud water in \cite{Srivastava1967} is generated simply,
%\begin{equation*}
%  P_c = -w\partd{q_v}{z};
%\end{equation*}
%we add terms to represent cloud water droplet activation via condensation nuclei,
%\begin{equation}\label{eq:qc_prod}
%  P_c = -w\partd{q_v}{z} + A(T, \Delta T, q_v,q_{aer}),
%\end{equation}
%where $q_{aer}$ is the mixing ratio of aerosol cloud condensation nuclei.
%\todo{Need some help with \eqref{eq:qc_prod}.}
%Rain production is decomposed into three subprocesses, $P_r=P_1 + P_2 + P_3$.
%The first subprocess represents evaporation in convective downdrafts,
%\begin{equation}
%  P_1 = -\mu(w) w \partd{q_v}{z}, \quad \mu(w) = \begin{cases} 1 & w<0, \\ 0 & w >= 0\end{cases}.
%\end{equation}
%The second subprocess represents cloud water conversion,
%\begin{equation}
%  P_2 = \begin{cases} 0 & q_c \le  q_c^{(crit)} \\ \alpha(q_c - q_c^{(crit)}) & q_c >  q_c^{(crit)}\end{cases},
%\end{equation}
%where $\alpha$ is a constant parameter representing the conversion time scale and $q_c^{(crit)}$ is an additional parameter, representing the critical value of cloud water, above which cloud water is converted to rain water.
%\pbc{Perhaps we should consider smoothing out these piecewise functions?}
%The last subprocess represents accretion, the capture of cloud drops by rain drops,
%\pbc{I assume that conversion implies $C: q_c,q_c \mapsto q_r$, whereas accretion implies $Ac:q_c,q_r\mapsto q_r$.}
%The model assumes that rainwater fall speed is proportional to droplet size, $W_r(D) = a D^b$, where $a$ and $b$ are constants, and that rain water droplet sizes are governed by the distribution function $N(D) = N_0e^{-\lambda} D$, where $\lambda = 3.67/D_0$ is a variable parameter related to $D_0$, the median diameter of rainwater droplets \cite[eqn.~(18)]{Srivastava1967}.
%With these assumptions, the expression for accretion is
%\begin{equation}
%  P_3 = \frac{\pi N_0 a \rho \Gamma(3+b) q_r}{4\lambda^{3+b}},
%\end{equation}
%where $\Gamma$ represents the Gamma Function (not a lapse rate).
%The same assumptions give the expression for rain fall velocity,
%\begin{equation}
%  w_r = \frac{a\Gamma(4+b)}{6\lambda^b}.
%\end{equation}
%
%Initial conditions for $w$ and $q_c$ are defined by solving the above equations with $q_r=0$, and $w(z_c)=q_c(z_c) = 0$, $T(z_c) = 277$ K, where $z_c$ is the height of the cloud base.
%\todo{Have to figure these out ourselves. Using standard thermodynamics and hydrostatic balance.}
%\pbc{I believe cloud base is the first point above $z=0$ where $q_c>0$.}
%The bottom boundary condition is $w(0) = 0$.
%At $z_{top} = 9.5$ km, if $w(z_top)>0$ then the equations are solved without regard for the boundary; if $w(z_top)<=0$ the other variables are held constant.
%
%%------------------- table: cloud model parameters ---------------
%\begin{table}[htbp]
%\centering
%\caption{Parameters of the 1D cloud model.}
%\label{tab:cloud_model_parameters}
%\begin{tabular}{ccccc}
%  \toprule
%  Parameter      &  value 1
%                 &  value 2
%                 & value 3\\
%  \midrule
%  $a$ 	 &  1500 cm$^{1/2}$/s  &  --  & -- \\
%  $b$    &  0.5                &  --  & --  \\
%  $\lambda$  &  1.0            &  --  & --  \\
%  $N_0$ &  80 (gm cm)$^{-1}$   &  --  & --  \\
%  $\alpha$ & 2.0E-5 s$^{-1}$ (14 hr) & 4.0E-5 s$^{-1}$ (7 hr) & 4.0E-5 s$^{-1}$ (4 min) \\
%  $q_c^{(crit)}$ & 0 & 3 gm/kg & 3 gm/kg \\
%  \bottomrule
%\end{tabular}
%\end{table}
%%-------------



\subsection{Embedded parameterization}

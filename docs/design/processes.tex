\chapter{Available Aerosol Processes}
\labelchapter{processes}

Haero offers implementations for each of the important stages in the aerosol
life cycle. Here we describe the physics of each stage, the approximations
made by the associated parametrizations, and the implementations of the
underlying prognostic and diagnostic processes.

In each of the processes, we assume the state of an aerosol system is specified
by the following prognostic variables:

\begin{itemize}
  \item $ n_m(x_i, y_i, z_k, t)$ [\# /m$^3$]: the {\em number density} of mode $m$,
        defined within the cell in column $i$ at vertical level $k$
  \item $ q_{m,s}(x_i, y_i, z_k, t)$ [kg aerosol species $s$/kg air]: the
        {\em mass mix fraction} of aerosol species $s$ within mode $m$, defined
        within the cell in column $i$ at vertical level $k$
  \item $ q_g(x_i, y_i, z_k, t)$ [kmol gas species $g$/mkol air]: the {\em mole
        fraction} of gas species $g$, defined within the cell in column $i$ at
        vertical level $k$
\end{itemize}

Further, each process is independent of all other processes. In other words,
state variables are taken directly from the prognostics and diagostics passed
to the process, and no assumptions are made about how these state variables
were computed.

\begin{assume}[Independent aerosol processes]
  Every aerosol process is independent of other aerosol processes. Specifically:
  a process takes a set of state variables and diagnostics, and {\em using only
  these variables}, performs one of the following procedures, based on its
  process type:
  \begin{enumerate}
    \item If the process is {\em prognostic}, it computes a set of tendencies
          for the prognostic state variables at simulation time $t$.
    \item If the process is {\em diagnostic}, it updates any relevant diagnostic
          variables in place at simulation time $t$.
  \end{enumerate}
\end{assume}

This process independence is a dramatic departure from prior implementations of
aerosol physics in MAM. Any sequential coupling of processes must be performed
by a host model.

With this assumption, it's possible to recover the legacy MAM algorithms by
constructing a coupling procedure that follows the relevant assumptions. But
it's also possible to construct more sophisticated coupling processes that take
advantage of advances in numerical analysis.

\section{Nucleation}
\labelsection{nucleation}

Aerosol nucleation refers to the formation of new aerosol particles from gas
molecules. Nucleation is an important source of aerosol particles in the
atmosphere, and can have a significant effect on other aerosol-related
processes. As a conversion mechanism, it reduces the population of gas-phase
species and increases aerosol mass and number concentrations.

Aerosol nucleation can also affect the formation of clouds and fog. The newly
formed particles that result from nucleation are generally hydrophilic and not
efficiently scavenged because of their small sizes. These particles can
grow larger and act as nuclei for cloud condensation.

Haero currently offers one process implementation for nucleation: the legacy
MAM4 model, written in Fortran.

\subsection{Nucleation Physics}
\labelsubsection{nuc:physics}

In the atmosphere, gas molecules exist either independently or in small
polymolecular clusters~\cite{seinfeld-2006-acp}. In general, the concentration
of independent gas molecules is much higher than that of clusters. Therefore,
the radius of a cluster is changed mainly due to the collision between an
independent gas molecule and a cluster. If the radius of a cluster is larger
than $r^*$, the radius of the so-called ``critical cluster'' or nucleus, this
cluster tends to grow rapidly and nucleation happens; otherwise, this cluster
tends to shrink and nucleation does not occur.

Nucleation can occur {\em homogeneously} (in the absence of pre-existing
particles) or {\em heterogeneously} (with pre-existing particles present).
In either case, nucleation can involve a single species or more than one species.
Based on these circumstances, we can classify any nucleation process:

\begin{itemize}
  \item {\bf Type 1} nucleation is homogeneous and involves a single species.
  \item {\bf Type 2} nucleation is homogeneous and involves two or more species,
        e.g. water vapor ($\rwater$) and sulfuric acid gas ($\rsag$).
  \item {\bf Type 3} nucleation is heterogeneous and involves a single species.
        A good example is the formation of water droplets in the presence of
        pre-existing particles.
  \item {\bf Type 4} nucleation is heterogeneous and involves two or more
        species.
\end{itemize}

In all four cases, the nucleation process alters the vapor mixing ratios and
aerosol mass and number mixing ratios. Their rates of change are

\begin{align}
  \ddt{q_{m,s}} &= \ddt{q_{n,s}} \cdot \frac{m_{d,s}}{\mu_{s}} \,, \labeleq{nuc:aer}\\
  \ddt{q_g}     &= -\ddt{q_{m,s}} \,, \labeleq{nuc:gas}
\end{align}

where
\begin{itemize}
  \item $q_{m,s}$ [kmol aerosol species $s$/ kmol dry air] is the aerosol mass
        mixing ratio of species $s$
  \item $q_{n,s}$ [\# aerosol species $s$/ kmol dry air] is the aerosol number
        mixing ratio for species $s$
  \item $q_g$ [kmol gas species $g$/ kmol dry air] is the mole fraction for gas
        species $g$
  \item $m_{d,s}$ [kg] is the dry mass of a nucleus belonging to aerosol species
        $s$
  \item $\mu_{s}$ [kg/kmol] is the molecular weight of aerosol species $s$
\end{itemize}

% key assumptions
\subsection{MAM4 Nucleation Process}
\labelsubsection{nuc:mam4}

Although we have mentioned four types of aerosol nucleation, the MAM4 nucleation
process treats only type 2. MAM4 makes several assumptions on top of the modal
aerosol assumption to arrive at its governing equations for nucleation:

\begin{assume}[MAM4 Nucleation: Aitken mode]
  There is no nucleation mode---instead, the Aitken mode is used to capture the
  smallest nucleated aerosol particles.
\end{assume}

Previous research~\cite{kerminen-2002-jas} suggests that the typical size
range of newly formed nuclei are around 1 nm in diameter.

\begin{assume}[MAM4 Nucleation: new nucleus size]
  Newly formed nuclei have a 1 nm diameter.
\end{assume}

Particles of this size are too small to be represented directly by MAM4's Aitken
mode. To reconcile this difficulty with its modal approximation, the MAM4
process breaks nucleation into two distinct stages.

\begin{assume}[MAM4 Nucleation: separate creation and growth stages]
  The nucleation process consists of two stages:
  \begin{enumerate}
    \item Fresh nuclei are formed from the vapors of gas species.
    \item These nuclei grow to a predefined size that falls within the range
      of the Aitken mode. This predefined size is described in
      \refparagraph{nuc:growth_nuclei}.
  \end{enumerate}
  Therefore, nucleation affects only the aerosol number and mass mixing ratios
  in the Aitken mode.
\end{assume}

\begin{assume}[MAM4 Nucleation: gas/aerosol density]
  In the MAM4 nucleation process, the mass densities of aerosol sulfuric acid
  and sulfate are assumed to be the same: $\rm 1770~kg~m^{-3}$
\end{assume}

\begin{assume}[MAM4 Nucleation: averaged $\rsag$ number concentration]
  To calculate the number concentration of $\rsag$ [\# / cc air], we use
  $c\dsub{n,\sag}$, the average mass mixing ratio of $\rsag$ over the
  condensation process. We use $c\dsub{n, \sag}$ instead of the mass mixing
  ratio of $\rsag$ given at the beginning of the nucleation process.

  $$c\dsub{n,\sag} = 10^{-3} \cdot N_A \cdot \bar{q_{\sag}} c\dsub{air}$$

  where $N_A$ is Avogadro's constant and $c\dsub{air}$ [kmol air / m$^3$ air] is
  the air molar concentration.
\end{assume}

\begin{assume}[MAM4 Nucleation: homogeneous only]
  MAM4 treats only the homogeneous nucleation of $\rsag \mhyphen \rwater$
  (type 2).
\end{assume}

With these assumptions, the governing equations of MAM4's nucleation process
are

\begin{align}
  \frac{dq\dsub{n,i}}{dt} &= \frac{10^6 J\dsub{nuc}}{c\dsub{air}} \,, \text{ i = 2 for Aitken mode} \,, \labeleq{nuc:jnuc} \\
  \frac{d\amass{\asf}}{dt} &= \frac{dq\dsub{n,i}}{dt} \cdot \frac{m\dsub{d,\asf}}{\mu_{\asf}} \,, \text{ i = 2} \,, \labeleq{nuc:amass} \\
  \frac{d\vmass{\sag}}{dt} &= -\frac{d\amass{\asf}}{dt} \,, \text{ i = 2} \,, \labeleq{nuc:vmass}
\end{align}

where $J\dsub{nuc}$ [\# / (cc $\cdot$ s)] is the nucleation rate describing the
net number of clusters that grow past $r^*$ per unit time per cm$^{3}$ of
air. $J\dsub{nuc}$ depends on the available amount of gas species, temperature
and relative humidity.

% parameterization
\subsubsection{MAM4 nucleation parameterizations} \labelsubsubsection{nuc:paranuc}

In order to solve the above governing equations, one first computes $J\dsub{nuc}$,
which appears in \refeq{nuc:jnuc}. In principle, $J\dsub{nuc}$ can be calculated
thermodynamically by classical nucleation theory, which is described in
Chapter 11 of~\cite{seinfeld-2006-acp}. However, this classical treatment
carries a heavy computational cost and is not used in the MAM4 model. Instead,
we devise parameterizations to obtain $J\dsub{nuc}$.

Here we focus on the parameterizations of binary nucleation of
$\rsag \mhyphen \rwater$ used in MAM4.

The following notation simplifies our presentation:
\begin{itemize}
  \item RH: clear-sky relative humidity.
  %\item $c\dsub{n,\sag}$ [\# / cc air]: number concentration of $\rsag$.
  \item $n\dsub{\sag}$ [\#]: number of $\rsag$ molecules in a critical cluster.
  \item $n\dsub{tot}$ [\#]: number of molecules in a critical cluster.
  \item $J^*$ [\# / (cc $\cdot$ s)]: the {\em intermediate nucleation rate},
        which is modified by a sequence of parameterizations
        (\refparagraph{nuc:binary} -~\refparagraph{nuc:growth_nuclei}) to
        obtain $J\dsub{nuc}$.
  \item $C\dsub{sum,\sag}$ [s$^{-1}$]: the sum of mass transfer coefficients of
        $\rsag$ over the $I$ modes during the condensation process. This is a
        constant for the nucleation process.
\end{itemize}

With this assumption, the following nucleation parameterizations are evaluated
in order. First, RH is bounded within [0.01, 0.99] since the nucleation process
in MAM4 is only considered for clear-sky conditions. In clouds, $\rsag$
rapidly transfers into cloud droplets, and nucleation is unlikely to occur.

% binary nucleation parameterization by vk2002
\paragraph{Binary nucleation of $\rsag \mhyphen \rwater$}
\labelparagraph{nuc:binary}

MAM4 uses the binary parameterization developed by
Vehkam\"aki et al.~\citep{vehkamaki-2002-jgr}. This parameterization fits $J^*$
to a polynomial function of $T$, $RH$ and $c\dsub{n,\sag}$. This representation
is valid under the following conditions:

\begin{itemize}
  \item $T \in [230.15, 305.15]$;
  \item $RH \in [0.0001, 1.0]$;
  \item $c\dsub{n,\sag} \in [10^4, 10^{11}]$;
  \item $J^* \in [10^{-7}, 10^{10}]$;
\end{itemize}

Only the \textbf{first three} constraints are actually applied in the MAM4
process. $T$ and $RH$ are bounded within the given range. If
$c\dsub{n,\sag} \le 10^4$ or $\vmass{\sag,avg} \le 4 \cdot 10^{-16}$
(two roughly equivalent conditions), nucleation is assumed to not occur,
$J\dsub{nuc}$ is set to $0$ in \refeq{nuc:jnuc}, and the calculation
terminates.

If the above conditions are satisfied, this parameterization computes the
following quantities in order:

\begin{enumerate}
  \item The mole fraction of $\rsag$ in a critical cluster (i.e., $x^*$) is
        given by:
    \begin{align}
      x^* &= 0.740997 - 0.00266379 T - 0.00349998 \ln c\dsub{\sag}
             + 0.0000504022 T \ln c\dsub{\sag} \nonumber \\
          &+ 0.00201048 \ln RH - 0.000183289 T \ln RH
             + 0.00157407 (\ln RH)^2 \nonumber \\
          &- 0.0000179059 T (\ln RH)^2 + 0.000184403 (\ln RH)^3 \nonumber \\
          &- 1.50345 \cdot 10^{-6} T (\ln RH)^3 \,.
    \end{align}
  \item The intermediate nucleation rate (i.e., $J^*$) is calculated from
        $x^*$:
        \begin{align}
          J^* &= \exp \Big[ a(T,x^*) + b(T,x^*) \ln RH +
                 c(T,x^*) (\ln RH)^2 + d(T,x^*) (\ln RH)^3 \nonumber \\
              &+ e(T,x^*) \ln c\dsub{\sag} + f(T,x^*) (\ln RH) \ln c\dsub{\sag}
                 + g(T,x^*) (\ln RH)^2 \ln c\dsub{\sag} \nonumber \\
              &+ h(T,x^*) (\ln c\dsub{\sag})^2 + i(T,x^*) (\ln RH)
                (\ln c\dsub{\sag})^2 + j(T,x^*) (\ln c\dsub{\sag})^3 \Big] \,, \labeleq{vk2002_nucrate}
        \end{align}
        where the coefficients $a(T,x^*) \dots j(T,x^*)$ are given in
        \refappendixsubsection{nuc:coeff_nucrate}.
  \item $n\dsub{tot}$, the total number of molecules in a critical cluster,
        is calculated from $x^*$:
        \begin{align}
          n\dsub{tot} &= \exp \Big[ A(T,x^*) + B(T,x^*) \ln RH
                         + C(T,x^*) (\ln RH)^2 + D(T,x^*) (\ln RH)^3 \nonumber \\
                      &+ E(T,x^*) \ln c\dsub{\sag} + F(T,x^*) \ln RH \ln c\dsub{\sag}
                         + G(T,x^*) (\ln RH)^2 \ln c\dsub{\sag} \nonumber \\
                      &+ H(T,x^*) (\ln c\dsub{\sag})^2
                       + I(T,x^*) \ln RH (\ln c\dsub{\sag})^2
                       + J(T,x^*) (\ln c\dsub{\sag})^3 \Big] \,, \labeleq{vk2002_ntot}
        \end{align}
        where the coefficients $A(T,x^*) \dots J(T,x^*)$ are given in
        \refappendixsubsection{nuc:coeff_ntot}.
  \item $n\dsub{\sag}$, the total number of $\rsag$ molecules in a critical
        cluster, is calculated from $x^*$ and $n\dsub{tot}$:
        \begin{equation}
          n\dsub{\sag} = n\dsub{tot} x^* \,.
        \end{equation}
  \item $r^*$ [nm], the radius of a critical cluster, is calculated from
        $x^*$ and $n\dsub{tot}$:
        \begin{equation}
          r^* = \exp \left[ -1.6524245 + 0.42316402 x^* +
                0.3346648 \ln n\dsub{tot} \right] \,.
        \end{equation}
\end{enumerate}

% binary nucleation parameterization by wang2009
\paragraph{Nucleation within the planetary boundary layer} \labelparagraph{nuc:pbl}

If nucleation occurs within the planetary boundary layer (PBL) or below a
a height of 100 meters, $J^*$ is compared with $J\dsub{PBL}$, an additional
parameterization that uses an empirical first-order estimation
~\cite{sihto-2006-acp,wang-2009-acp}. Here we make an additional assumption:

\begin{assume}[MAM4 Nucleation: new nucleus composition]
  New aerosol nuclei consist entirely of $\rsag$.
\end{assume}

The empirical estimation is
\begin{equation}
  J\dsub{PBL} = 10^{-6}~c\dsub{\sag} \,.
\end{equation}

If $J\dsub{PBL} > J^*$, the the following adjustments are made:
\begin{align}
  J^* &= J\dsub{PBL} \,, \\
  r^* &= 0.5~nm \,, \\
  n\dsub{\sag} &= \frac{6.023 \times 10^{23} \pi (1~nm)^3
                  \rho\dsub{\sag}}{6 \mu_{\sag}} \,, \\
  n\dsub{tot} &= n\dsub{\sag} \,.
\end{align}

Otherwise, $J^*$, $n\dsub{tot}$ and $n\dsub{\sag}$ remain the same values
calculated by the first parameterization in \refparagraph{nuc:binary}.

% parameterization of nuclei growth
\paragraph{Nuclei growth} \labelparagraph{nuc:growth_nuclei}

If $J^* < 10^{-6}$ based on the previous calculations, nucleation does not
occur.

\begin{assume}[MAM4 Nucleation: nucleation rate cutoff]
  If $J^*$, the intermediate nucleation rate, is less than $10^{-6}$ molecules
  per cc per second, based on binary nucleation and PBL effects, then no
  nucleation occurs.
\end{assume}

Otherwise, the nucleation calculation proceeds, and the following quantities are
computed:

\begin{align}
  RH &\in [0.1, 0.95] \,, \\
  f\dsub{v} &= 1 - \frac{0.56}{\ln RH} \,, \\
  V\dsub{d,nuc} &= \frac{n\dsub{\sag} \mu_{\asf}}
                   {6.023 \cdot 10^{26} \rho\dsub{\sag}} \,, \\
  D\dsub{d,nuc} &= \left( \frac{6 V\dsub{d,nuc}}{\pi} \right)^{\frac{1}{3}} \,,
\end{align}

where

\begin{itemize}
  \item $f\dsub{v}$ is the ratio of wet volume over dry volume of a particle,
        using a simple K\"ohler approximation for $\rm NH\dsub{4}HSO\dsub{4}$
  \item $V\dsub{d,nuc}$ [$\rm m^3$] $D\dsub{d,nuc}$ [m] are the dry volume and
        diameter of a nucleus, estimated from the binary nucleation and PBL
        parameterizations
\end{itemize}

The simplified approximation for $f\dsub{v}$ follows the K\"ohler theory used by
the MAM4 aerosol water uptake, but neglects surface curvature effects. The
composition is assumed to be pure ammonium bisulfate with hygroscopicity = 0.56.

To decide whether the nuclei have grown large enough to fit within the Aitken
mode's acceptible range of sizes, we use this mode's predefined min, max, and
mean particle diameters [m] to calculate two quantities:
\begin{align}
  D\dsub{p,lo} &= \exp \left[ 0.67 \ln (8.7 \cdot 10^{-9}) +
                            0.33 \ln (26 \cdot 10^{-9}) \right] \,, \\   %% about 12.5 nm
  D\dsub{p,hi} &= 52 \cdot 10^{-9} \,.
\end{align}
\textcolor{red}{Jian: Dick said $D\dsub{p,lo}$=8.7 nm might make more sense.}

If $D\dsub{d,nuc} > D\dsub{p,lo}$, $J\dsub{nuc}$ is set to $J^*$. Otherwise,
the nuclei are too small for the Aitken mode. We use the following
parameterization to consider the growth of nuclei to a larger size
\cite{kerminen-2002-jas}. This parameterization makes the following assumptions:

\begin{assume}[MAM4 Nucleation: no self-coagulation]
  The only significant ``sink'' mechanism for new aerosol nuclei is their
  coagulation with pre-existing particles. In other words, the MAM4 model
  ignores the self-coagulation of fresh aerosol nuclei.
\end{assume}

\begin{assume}[MAM4 Nucleation: growth by constant condensation]
  The only important source for the growth of nuclei is the condensation
  of vapors, and this condensation rate is assumed to be constant.
\end{assume}

\begin{assume}[MAM4 Nucleation: static growth process]
  The population of pre-existing particles and the concentration of
  condensible vapors remain unchanged during the growth process for new
  nuclei.
\end{assume}

\begin{assume}[MAM4 Nucleation: non-volatile vapors]
  The condensable vapors responsible for nuclei growth are non-volatile.
\end{assume}

Using this parameterization, we adjust $J^*$ based on

\begin{itemize}
  \item the increase of the size of nuclei due to condensation of $\rsag$
  \item the decrease of the number concentration of nuclei due to
        coagulation with pre-existing particles
\end{itemize}

The following quantities are calculated in the following order for the
adjustment of $J^*$:

\begin{align}
  v\dsub{\sag} &= 14.7 \sqrt{T} \,, \\
  \rho\dsub{nuc} &= \frac{\rho\dsub{\sag}}{f\dsub{v}} \,, \\
  GR &= \frac{3.0 \cdot 10^{-9} v\dsub{\sag} \mu_{\sag} c\dsub{n,\sag}}
        {\rho\dsub{nuc}} \,, \\
  D\dsub{nuc,ini} &= \max (2r^*, 1) \,, \\
  D\dsub{nuc,fin} &= 10^9 D\dsub{p,lo} f\dsub{v}^{\frac{1}{3}} \,, \\
  \gamma &= 0.23 D\dsub{nuc,int}^{0.2}
                 (\frac{D\dsub{nuc,fin}}{3})^{0.075}
                 (\frac{\rho\dsub{nuc}}{1000})^{-0.33}
                 (\frac{T}{293})^{-0.75} \,, \labeleq{nuc:gamma} \\
  \mathbb{D}\dsub{g,\sag} &= \frac{6.7037 \cdot 10^{-9} T^{0.75}}
                             {c\dsub{air}} \,, \\
  CS^\prime &= \frac{C\dsub{sum,\sag}}
               {4 \pi \mathbb{D}\dsub{g,\sag} \alpha\dsub{\sag}} \,, \label{eq:cs_prime} \\
  \eta &= \frac{\gamma CS^\prime}{GR} \,, \\
  J\dsub{nuc} &= J^* \exp \left( \frac{\eta}{D\dsub{nuc,fin}} -
                     \frac{\eta}{D\dsub{nuc,ini}} \right) \,,
\end{align}

where
\begin{itemize}
  \item $v\dsub{\sag}$ [m/s] is the approximated mean molecular speed of $\rsag$
  \item $\rho\dsub{nuc}$ [kg/m$^3$] is the mass density of nuclei after
        accounting for aerosol water uptake by sulfate
  \item $GR$ [m/s] is the nuclei growth rate, assumed to be constant
  \item $D\dsub{nuc,ini}$ [nm] and $D\dsub{nuc,fin}$ [nm] are the wet diameters
        of nuclei before and after growth
  \item $\mathbb{D}\dsub{g,\sag}$ [$\rm m^2/s$] is the approximate gas
        diffusivity of $\rsag$
  \item $CS^\prime$ is the ``condensation sink'', constant based on the
        assumption that the population of pre-existing particles does not
        change
\end{itemize}

The formula for $CS^\prime$ used in MAM4 comes from the condensation equation of
$\rsag$ gas and differs from that in~\cite{kerminen-2002-jas}. According to Dick
Easter, this difference is minor and produces no significant changes in
nucleation results.
% and the detailed derivation can be found in the Appendix~\ref{cs_prime}.
%Note that compared with the Eq. (22) in~\cite{kerminen-2002-jas},
%Dick also dropped a term in \refeq{nuc:gamma}, which was
%thought to be close to 1.

\subsubsection{MAM4 nucleation: numerical considerations}

After $J\dsub{nuc}(t\dsub{0})$ is calculated using the parameterizations in
\refsubsubsection{nuc:paranuc}, $\Delta \aitmass{\asf}$---the increase of
$\aitmass{\asf}$ due to nucleation---can be calculated:
%
\begin{align}
  \Delta \aitmass{\asf} &= \frac{10^6 J\dsub{nuc}(t\dsub{0}) \Delta t\dsub{nuc} m\dsub{d,\asf}}
                            {c\dsub{air}\mu_{\asf}} \,, \\
  m\dsub{d,\asf} &= \begin{cases}
    \rho\dsub{\sag} \frac{\pi}{6}V\dsub{p,hi}^3 \,, \, \text{if $D\dsub{d,nuc} \ge D\dsub{p,hi}$} \,, \\
    \rho\dsub{\sag} \frac{\pi}{6}V\dsub{d,nuc}^3 \,, \, \text{if $D\dsub{p,hi} > D\dsub{d,nuc} > D\dsub{p,lo}$} \,, \\
    \rho\dsub{\sag} \frac{\pi}{6}V\dsub{p,lo}^3 \,, \, \text{otherwise} \,,
  \end{cases}
\end{align}

where $\Delta t\dsub{nuc}$ = $t\dsub{1}$ - $t\dsub{0}$ = $\Delta t\dsub{phys}$.

%(See ``MAM\_Basics'' documentation for the value of $\Delta t\dsub{phys}$).
Since $\Delta \aitmass{\asf}$ cannot exceed the available amount of
$\vmass{\sag}(t\dsub{0})$, we apply a limiting factor $f^*$, computed
as the following fraction:

\begin{equation}
  f^* = \begin{cases}
        \frac{\vmass{\sag}(t\dsub{0})}{\Delta \aitmass{\asf}}, \, \text{if $\Delta \aitmass{\asf} > \vmass{\sag}(t\dsub{0})$} \,, \\
        1, \text{otherwise.} \\
  \end{cases}
\end{equation}

If $J\dsub{nuc} f^* < 10^{-18}$, $J\dsub{nuc}$ is set to $0$ and no nucleation
occurs. Otherwise, we make the following adjustments:

\begin{align}
  \Delta \aitmass{\asf} &= \min ( 0.9999\vmass{\sag}(t\dsub0), f^* \Delta \aitmass{\asf} ) \,, \\
  \Delta \vmass{\sag} &= - \Delta \aitmass{\asf} \,, \\
  \Delta q\dsub{n,2} &= \frac{\Delta \aitmass{\asf} \mu_{\asf}}{m\dsub{d,\asf}} \,.
\end{align}

If $\frac{\Delta q\dsub{n,2}}{\Delta t\dsub{nuc}} < 100$, no nucleation occurs.
Otherwise we continue with the following adjustments:

\begin{itemize}
  \item If $\frac{\Delta \aitmass{\asf} \mu_{\asf}}{\Delta q\dsub{n,2}} <
        \frac{\pi}{6} \rho\dsub{\asf} D\dsub{p,lo}^3$, the nuclei size is
        increased at the expense of the number concentration:
        $$\Delta q\dsub{n,2} = \frac{\Delta \aitmass{\asf} \mu_{\asf}}{\frac{\pi}{6} \rho\dsub{\asf} D\dsub{p,lo}^3}$$
  \item If $\frac{\Delta \aitmass{\asf} \mu_{\asf}}{\Delta q\dsub{n,2}} >
        \frac{\pi}{6} \rho\dsub{\asf} D\dsub{p,hi}^3$, the nuclei size is
        decreased, and mass is reduced accordingly:
        $$\Delta \aitmass{\asf} = \frac{\frac{\pi}{6} \rho\dsub{\asf} D\dsub{p,hi}^3 \Delta q\dsub{n,2}}{\mu_{\asf}}$$
\end{itemize}

\jjc{This can't be the whole story---neither the number concentration nor the
mass is altered above.}

Dick commented that 1) the first ``if'' condition revealed the fact that
there was not enough $\rsag$ gas for all the nuclei to grow to the size
$D\dsub{p,lo}$. Thus Dick arbitrarily reduced the number of nuclei to
increase the size of nuclei; 2) the second ``if'' condition will never
happen, so it is just a ``sanity check''.

After the adjustment above, we solve the governing equations
\refeq{nuc:jnuc}-\refeq{nuc:vmass} numerically:

\begin{align}
  q\dsub{n,2} (t\dsub{1}) &= q\dsub{n,2} (t\dsub{0}) + \Delta q\dsub{n,2} \,, \\
  \aitmass{\asf} (t\dsub{1}) &= \aitmass{\asf} (t\dsub{0}) + \Delta \aitmass{\asf} \,, \\
%\Delta q\dsub{\sag} &= \min \left( \Delta \aitmass{\asf}, \vmass{\sag} (t\dsub{0}) \right) \,, \\
  \Delta q\dsub{\sag} &= -\Delta \aitmass{\asf} \,, \\
  \vmass{\sag} (t\dsub{1}) &= \vmass{\sag} (t\dsub{0}) + \Delta q\dsub{\sag} \,.
\end{align}

\subsubsection{Additional MAM4 nucleation diagnostics} \label{output}

The MAM4 nucleation process provides the following diagnostic variables:

\begin{enumerate}
  \item $J^*$: the nucleation rate after the first and second (if applicable)
        parameterizations
  \item $q\dsub{n,2}$: the number mixing ratio of $\rasf$ in the Aitken mode at
        $t = t\dsub1$
  \item $\vmass{\sag} (t\dsub1)$: the mass mixing ratio of $\rsag$ at
        $t = t\dsub1$
  \item $\aitmass{\asf} (t\dsub1)$: the mass mixing ratio of $\rasf$ in the
        Aitken mode at $t = t\dsub1$.
\end{enumerate}

\jjc{Here's an example of a prognostic process that computes diagnostic
variables ``along the way.'' Should we discuss whether to make these diagnostics
available to other processes?}

\subsubsection{MAM4 nucleation verification tests} \labelsubsubsection{nuc:nuctests}

The parameterizations we've described above allow us to make some general
observations about output from the MAM4 process, given valid input:

\begin{itemize}
  \item All aerosol mix fraction tendencies are zero except those for the Aitken
        mode, which are zero or positive.
  \item All modal number density tendencies are zero except those for the Aitken
        mode, which are zero or positive.
  \item All gas-phase species tendencies are non-positive. Any positive
        aerosol tendencies in the Aitken mode are associated with negative
        tendencies in gas-phase species.
  \item All newly-formed aerosol nuclei are $\rsag$ and are 1 nm in diameter.
\end{itemize}

\jjc{Can we say anything about mass conservation between gases and aerosols?}

These conditions hold for any valid input. We can say more about output for
input with given characteristics:

%\begin{itemize}
%\end{itemize}



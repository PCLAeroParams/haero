\section{Governing Equations}
\labelsection{equations}

Here we offer an extremely abbreviated ``fly-over'' of the evolution equations
that quantify the dynamics of aerosols. Our discussion emphasizes the
mathematical representation of aerosols. For a more detailed explanation of
the underlying ideas, we refer the reader to~\cite{Whitby1991}
and~\cite{Friedlander1977}.

All quantities are in SI units unless specifically mentioned. To keep the
discussion simple and focused, we avoid referencing specific coordinates.

\subsection*{Size Matters}

MAM's approach to modeling aerosols is based on the observations that

\begin{itemize}
  \item the sizes of aerosol particles greatly influence their dynamical
        behavior
  \item these sizes span several orders of magnitude (from 0.001 to 100 microns)
\end{itemize}

Here, a ``particle'' is an individual aerosol molecule with some volume $V_p$.
Aerosol particles consist of polymers, and a particle consisting of a chain of
$i$ monomers is an $i$-mer.

How do we represent a set of aerosol particles in space, given the importance
of their size? In the simplest case (for small $i$-mers), we can denote
$N_i(\vec{x}, t)$ as the number of $i$-mers per cubic meter in the vicinity of
the point $\vec{x}$ at time $t$. In this language, the evolution equation is a
set of coupled advection-diffusion-reaction (ADR) equations acting under the
influence of a bulk velocity $\vec{v}$:

\begin{equation}\labeleq{small_imer_dNdt}
  \ddt{N_i} + \div{(N_i\vec{v})} = \mathcal{D}(\nabla N_i) +
                                   \mathcal{R}(\{N_j\}) +
                                   \mathcal{S}(N_i, t)
\end{equation}

Here we represent terms on the right hand side by

\begin{itemize}
  \item $\mathcal{D}$: terms related to {\it diffusion processes} for $N_i$
  \item $\mathcal{R}$: terms related to {\it reaction processes}, in which
        various $j$-mers combine, react, and dissociate to form particles of
        other types
  \item $\mathcal{S}$: terms related to {\it source and sink processes},
        including external and prescribed sources of $N_i$.
\end{itemize}

We describe the forms of these terms in \refsection{tendencies}. The bulk
velocity $\vec{v}$ is supplied by some dynamical atmospheric model.

For larger $i$-mers ($i > 100$), the above description becomes inadequate, and
we must adopt a description for particle numbers that admits a continuous
number distribution in particle size space.

Let $n(V_p, \vec{x}, t)$ be the number of particles of volume $V_p$ occupying
the point $\vec{x}$ at time $t$. Then the evolution of all $i$-mers is described
by a single ADR equation:

\begin{equation}\labeleq{continuous_dndt}
  \ddt{n} + \div{(n\vec{v})} = \mathcal{D}(\nabla n) +
                               \mathcal{R}(n) +
                               \mathcal{S}(n, t)
\end{equation}

where $\mathcal{D}$, $\mathcal{R}$, and $\mathcal{S}$ have been expressed
in terms of $n$ instead of $N_i$. This is a tidy equation, but $n$ is a
number distribution function that introduces a new dimension---the particle volume
$V_p$---to the solution space. This makes it inconvenient for doing numerical
calculations. In this description, $n$ can assume any shape in particle volume
space, which raises the question of how to constrain the solution in $V_p$.

\subsection*{Moment Equations and the Closure Problem}
We can reduce the size of our solution space by making a few assumptions:

\begin{itemize}
  \item \assume Aerosol particles are spherical, and their size can be
        parameterized by their diameter $D_p$.
  \item \assume The detailed structure of the number distribution function
        $n(V_p, \vec{x}, t)$ is unimportant to aerosol dynamics.
\end{itemize}

If we don't need to obtain the full solution for $n$, we can select a specific
functional form $n(D_p, \vec{x}, t) = n(\vec{x}, t; D_p)$ for it. Then, taking cues from
methods in particle kinetics and turbulence theory, we can integrate the product
of \refeq{continuous_dndt} with powers of $D_p$ to obtain the {\bf moment
equations}

\begin{equation}\labeleq{moments}
  \ddt{\mathcal{M}_k} + \div{(\mathcal{M}_k\vec{v})} = \int_0^{\infty} D_p^k (\mathcal{D} + \mathcal{R} + \mathcal{S}) \d{D_p}
\end{equation}

where $\mathcal{M}_k(\vec{x}, t) = \int_0^{\infty} D_p^k n(\vec{x}, t; D_p) \d{D_p}$
is the ``$k$th moment'' of $n$.

The moment equations are solved by picking a specific form of $n$'s functional
dependence on $D_p$ and playing tricks to avoid actually evaluating the
above integrals. However, the moment equations aren't closed---the advection,
diffusion, reaction, and source terms can all involve $n$ and its spatial
derivatives, and in general, the evolution of $\mathcal{M}_k$ is coupled to higher
moments. To make further progress, we must solve this {\bf closure problem}.

\subsection*{Modal Equations}

Having already given up on obtaining a general solution for $n(\vec{x}, t)$,
we allow ourselves to make another assumption:

\begin{itemize}
  \item \assume The number distribution function $n$ is the sum of a set of
        specific number distribution functions $n_i$, each representing a
        {\bf mode} with a specific functional form for a sub-population of
        aerosol particles occupying a certain range in particle size space.
\end{itemize}

This is the {\bf modal assumption}. In other words,

\begin{equation}\labeleq{modal_n}
  n(\vec{x}, t; D_p) = \sum_{i=1}^M n_i(\vec{x}, t; D_p)
\end{equation}

where $n_i$ represents aerosol particles with sizes falling within the range
of mode $i$ and $M$ is the number of modes. Each mode assumes a specific
functional form for $n_i$ in terms of its relevant size as given by $D_p$.
We include the arguments $\vec{x}$ and $t$ to emphasize that this equation holds
at each point in space and at each instant in time for an aerosol system.

When we adopt the modal assumption, we express our solution for particles in
mode $i$ in terms of its zeroth, first, second, and third moments
$\mathcal{M}_k^{(i)}(\vec{x}, t)$. These are, respectively:

\begin{itemize}
  \item $\mathcal{M}_0^{(i)} = N_i$, the total number of concentration for particles
        in mode $i$
  \item $\mathcal{M}_1^{(i)} = N_i\overline{D}_i$, the number-weighted mean diameter of
        particles in mode $i$
  \item $\mathcal{M}_2^{(i)} = \frac{1}{\pi}N_i\overline{A}_i$, a number-weighted
        surface area of particles in mode $i$
  \item $\mathcal{M}_3^{(i)} = \frac{6}{\pi}N_i\overline{V}_i$, a number-weighted
        volume of particles in mode $i$
\end{itemize}

In this language, the aerosol evolution equation for the $k$th moment of the
$i$th mode is

\begin{equation}\labeleq{modal_evolution}
  \ddt\mathcal{M}_k^{(i)} = F_k^{(i)}(N_i, \overline{D}_i, \overline{A}_i, \overline{V}_i, T, p, \mathsf{...})
\end{equation}

where the air temperature $T$ and the air pressure $p$ enter the moment
equations, via right hand side terms in \refeq{moments}. These are the
{\bf modal equations}.

\begin{itemize}
  \item \assume The time evolution of a moment $\mathcal{M}_k^{(i)}$ can be
    expressed in terms of the quantities $N_i$, $\overline{D}_i$,
    $\overline{A}_i$, $\overline{V}_i$, the air temperature $T$, the pressure $p$, and a few
    selected additional quantities.
\end{itemize}

Solving the closure problem is then reduced to finding expressions for
$\overline{D}_i$, $\overline{A}_i$, and $\overline{V}_i$ that relate the
different moments of $n_i$:

\begin{align}
  \overline{D}_i &= \overline{D}_i(\mathcal{M}_{k_1}^{(i)}/N_i, \mathcal{M}_{k_2}^{(i)}/N_i) \labeleq{D_i}\\
  \overline{A}_i &= \overline{A}_i(\mathcal{M}_{k_1}^{(i)}/N_i, \mathcal{M}_{k_2}^{(i)}/N_i) \labeleq{A_i}\\
  \overline{V}_i &= \overline{V}_i(\mathcal{M}_{k_1}^{(i)}/N_i, \mathcal{M}_{k_2}^{(i)}/N_i) \labeleq{V_i}
\end{align}

Here, we formulate the above expressions by the selecting pairs of moments that
contribute to the evolution of each quantity for mode $i$.

\begin{itemize}
  \item \assume The averaged quantities $\overline{D}_i$, $\overline{A}_i$,
    and $\overline{V}_i$ can each be expressed as a function of a pair of
    specific number-normalized moments for the $i$th mode.
\end{itemize}

\subsection*{Log-Normal Distribution Functions}

MAM constructs its multi-modal distribution functions from log-normal
probability distribution functions (PDFs) for each mode $i$. The PDF for mode
$i$ expresses the fraction of aerosol particles present per unit size interval,
in terms of a continuous particle diameter $D_p$ within that mode. Such PDFs
have been found to represent measured aerosol size distributions with a level of
accuracy comparable to that of the relevant measurement
techniques~\cite{Whitby1991}.

\begin{itemize}
  \item \assume The probability distribution function $f_i$ used to construct
        the number distribution function $n_i$ for mode $i$ is a continuous
        function of the particle diameter $D_p$.
\end{itemize}

The log-normal distribution of mode $i$ in MAM's formalism is
\begin{equation}\labeleq{log_normal_pdf}
  f_i(D_p) = \frac{1}{\sqrt{2\pi} D_p \ln \sigma_{g,i}} \ 
      \exp \left [-\frac{(\ln D_p - \ln D_{g,i})^2}{2\ln^2 \sigma_{g,i}} \right]
\end{equation}
where we have introduced the geometric mean $D_{g,i}$ and the geometric standard
deviation $\sigma_{g,i}$ of $D_p$ within mode $i$. These two parameters are
determined for each mode.

This PDF can also be expressed in ``logarithmic form'' in terms of $\ln D_p$
of $D_p$:
\begin{equation}\label{eq:log_normal_pdf_log}
  g_i(\ln D_p) = \frac{1}{\sqrt{2\pi} \ln\sigma_{g,i}} \
      \exp \left [ -\frac{(\ln D_p - \ln D_{g,i})^2}{2\ln^2 \sigma_{g,i}} \right]
\end{equation}

Here, $g_i$ is a normal distribution of $\ln D_p$ with a mean of $\ln D_{g,i}$
and standard deviation of $\ln\sigma_{g,i}$.

These PDFs are referred to as {\bf normalized size distribution functions} in
MAM. We can justify this name by multiplying each PDF by the total particle
number density $N_i$ for mode $i$ to obtain the equivalent number distribution
functions:

\begin{align}\label{eq:log_normal_n}
  n_i(D_p) = N_i f_i(D_p) &= \frac{N_i}{\sqrt{2\pi} D_p \ln \sigma_{g,i}} \ 
  \exp \left [ - \frac{(\ln D_p - \ln D_{g,i})^2}{2\ln^2\sigma_{g,i}} \right ] \\
  \hat{n}_{i}(\ln D_p) = N_i g_i(\ln D_p) &= \frac{N_i}{\sqrt{2\pi}\ln \sigma_{g,i}} \ 
	\exp \left [ - \frac{(\ln D_p - \ln D_{g,i})^2}{2\ln^2\sigma_{g,i}} \right ]
\end{align}

The logarithmic form $\hat{n}_i$ is useful because of the relationship
$ n_i(D_p) = \hat{n}_i(\ln D_p)/D_p$, which allows us to write
\begin{equation} \labeleq{equ_norm_lognorm}
  n_i(D_p) \d{D_p} = \hat{n}_i(\ln D_p) \d{\ln D_p}
\end{equation}
to simplify integrands involving number distribution functions.

The choices of size range and width for the modes in MAM are based on
measurements of tropospheric aerosols (see~\cite{Easter2004} and references).
Table~\ref{tab:mode_size_parameters} shows the relevant parameters for MAM4,
the 4-mode legacy MAM model.

%------------------- table: mode parameters ---------------
\begin{table}[htbp]
\centering
%\caption{
%Geometric mean dry diameter ($D\dsub{gn,d,i}$, unit:
%$\rm \mu m$) and geometric standard deviation ($\sigma\dsub{g,i}$)
%of the number size distribution of each log-normal mode.
%}
%\begin{tabular}{ccccc}
%  \toprule
%  Mode        &  Lower bound of $D\dsub{gn,d,i}$  &  $D\dsub{gn,d,i}$   &
%  Upper bound of $D\dsub{gn,d,i}$  &  $\sigma\dsub{g,i}$ \\
%  \midrule
%  Aitken 	            &  0.0087  &   0.026   &   0.052  &  1.6 \\
%  Accumulation     &  0.0535  &   0.11      &   0.44    &  1.8 \\
%  Coarse               &  1.0        &   2.0       &   4.0      &  1.8 \\
%  Primary carbon  &  0.01      &   0.05      &   0.1      &  1.6 \\
%  \bottomrule
%\end{tabular}
\caption{Parameters of log-normal modes in MAM4.}
\label{tab:mode_size_parameters}
\begin{tabular}{ccccc}
  \toprule
  Mode           &  Lower bound of $D\dsub{gn,d,i}$
                 &  Upper bound of $D\dsub{gn,d,i}$  &  $\sigma\dsub{g,i}$ \\
  \midrule
  Aitken 	 &  0.0087  &   0.052  &  1.6 \\
  Accumulation   &  0.0535  &   0.44   &  1.8 \\
  Coarse         &  1.0     &   4.0    &  1.8 \\
  Primary carbon &  0.01    &   0.1    &  1.6 \\
  \bottomrule
\end{tabular}
\end{table}
%-------------

Thus, in MAM, all modes have the same mathematical form, but the parameters
are different for each mode.

\subsection*{Multi-Species Modes}

We have derived the modal equations assuming that each mode contains a
population of undifferentiated aerosol particles. If we wish to track individual
particle species within a mode, we may do so by expressing the population of a
mode as the sum of its constituent species in the modal assumption, adding
a species index $s$ to the number distribution function $n$:

\begin{equation}\labeleq{modal_multi_species}
  n(\vec{x}, t; D_p) = \sum_{i=1}^M \sum_{s=1}^S n_{i,s}(\vec{x}, t; D_p)
\end{equation}

with $S$ as the number of species. Then we can carry out the subsequent analysis
as before, breaking up the moments into species-specific parts
$\mathcal{M}_k^{(i,s)}$ and expressing the total $k$th moment as the sum of these
parts:

\begin{equation}
  \mathcal{M}_k^{(i)} = \sum_{s=1}^S \mathcal{M}_k^{(i,s)}
\end{equation}

Then the multi-species modal evolution equations are

\begin{equation}\labeleq{modal_multi_species_evolution}
  \ddt\mathcal{M}_k^{(i,s)} = F_k^{(i,s)}(N_{i,s}, \overline{D}_{i,s}, \overline{A}_{i,s}, \overline{V}_{i,s}, T, p, \mathsf{...})
\end{equation}

with species-specific equivalents of the quantities in \refeq{modal_evolution}.


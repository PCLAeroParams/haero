\chapter{Driver Input Specification}
\labelappendix{driver_input}

Below is the input specification for the Haero driver. The spec uses
[YAML](https://yaml.org/). Compared to Fortran namelists, YAML offers

\begin{itemize}
  \item {\bf better readability}: YAML files are flexible, and don't have a lot
    of "kibble" (braces, tags, and other stuff you see routinely in fancier
    markup like XML, HTML, and so on). They're very easy to read.
  \item {\bf better validation}: YAML files have named entries that can be
    used or ignored, depending on the needs of the particular application.
    Also, the file format allows for error checking.
  \item {\bf better language support}: Fortran namelists are so-called because
    they are only available to Fortran. YAML has libraries that allows it to
    be used with several programming languages (including Fortran!). This makes
    it easier for tools and other applications to use similar input files. In
    particular, this allows workflows to use a single input file to define a
    workload that can be processed with several tools.
  \item {\bf support for lists and maps}: YAML offers constructions for
    dynamically-sized datasets. Haero focuses on runtime configurability,
    so the ability to add species (and perhaps even modes) to an input file
    is important.
  \item {\bf a larger support community}: Fortran remains a niche language,
    which often instills a sense of pride in scientists. Unfortunately, the
    downside to belonging to a small community of specialists is that the tools
    are invariably of lower quality than more commonly-used tools. One needs
    only to witness the notorious ongoing issues with Fortran compilers to see
    this phenomenon firsthand.
\end{itemize}

This specification is for use only with the Haero driver. The library provides
no input/output interface---it assumes that the model in which it's embedded
does the work of assembling input data and loading it into the appropriate
data structures.

The Haero driver is a single-column aerosol dynamics simulator---it doesn not
model horizontal dynamics. However, depending on how initial conditions are
specified, it's possible to generate ensembles whose members are individual
atmospheric columns. Because the columns are independent of one another, these
ensembles can be run entirely within a single simulation, allowing the driver
to take advantage of parallelism, and allowing statistics and measures of
convergence to be generated.

\subsection*{Sections}

A YAML file consists of several named sections. Each of these sections can
contain data and metadata. Sections are a powerful tool for organizing input
using simple concepts with human-readable notation.

\subsubsection*{Modes}

The \verb modes  section defines the particle size modes available to a Haero
model. As we discussed in \refsubsection{lib:modes}, a mode has metadata
specifying its size range and its geometric standard deviation.

\begin{verbatim}
modes:
  aitken:
    D_min: 0.0087
    D_max: 0.052
    sigma: 1.6
  accumulation:
    D_min: 0.0535
    D_max: 0.44
    sigma: 1.8
  coarse:
    D_min: 1.0
    D_max: 4.0
    sigma: 1.8
  primary_carbon:
    D_min: 0.01
    D_max: 0.1
    sigma: 1.6
\end{verbatim}

Particle diameters and $\sigma$ are measured in $\mu\mathrm{m}$.

The \verb modes  section is essentially a map whose keys are mode names
(\verb aitken , \verb accumulation , \verb coarse , and \verb primary_carbon ,
in the example above), and whose values are themselves maps. The map for a given
mode contains the following fields:

\begin{itemize}
  \item \verb D_min : the minimum diameter of particles belonging to the mode
  \item \verb D_max : the maximum diameter of particles belonging to the mode
  \item \verb sigma : the geometric mean standard deviation for the size
                      distribution for particles belonging to the mode
\end{itemize}

Everything in a mode must be completely specified---there are no default values.

\subsubsection*{Aerosol Species}

Aerosol species populate each of the defined modes, and are defined in the
\verb aerosols  section. As we discussed in \refsubsection{lib:species}, a
species is defined by its elemental composition and its electric charge (given
in units of the electronic charge $|e|$).
%The \verb species  section follows the same format as defined by
%\href{https://cantera.org/documentation/dev/sphinx/html/yaml/species.html}{Cantera}.

Like the \verb modes  section, the \verb aerosols  section is a map whose keys
are {\em symbolic names} of aerosol particle species (e.g. \verb SO4  for
sulfate particles):

\begin{verbatim}
aerosols:
  SO4:
    name: sulfate
  NH4:
    name: ammonium
  POA:
    name: primary organic aerosol
  SOA:
    name: secondary organic aerosol
  BC:
    name: black carbon
  SS:
    name: sea salt
  DST:
    name: dust
  MOA:
    name: marine organic aerosol
\end{verbatim}

An aerosol species has the following fields:

\begin{itemize}
  \item \verb name : the full name for the species (e.g. \verb sulfate  for
    \verb SO4 ). You don't need to quote the full name, even if
                     it contains spaces.
\end{itemize}

\subsubsection*{Gas Species}

Gas species exist in the atmosphere, and aren't associated with modes. These
species are defined in the \verb gases  section.

The \verb gases  section is identical in structure to the
\verb aerosols section:

\begin{verbatim}
gases:
  SO2:
    name: sulfur dioxide
  H2SO4:
    name: sulfuric acid
  SOAG:
    name: single semi-volatile organic gas-phase species
  NH3:
    name: ammonia
\end{verbatim}

A gas species has the following fields:

\begin{itemize}
  \item \verb name : the full name for a gas species (e.g.
                     {\verb sulfur dioxide} for \verb SO4 . You don't need to quote the
                     full name, even if it contains spaces.
\end{itemize}

\subsubsection*{Physics}

In this section, we tell the Haero driver what physical processes we wish
to simulate. Every field in this section assumes a \verb true  or
\verb false  value, so it's really just a set of ON/OFF switches. Valid fields
are:

\begin{itemize}
  \item \verb growth : enables particle growth modeling
  \item \verb gas_chemistry : enables the gas chemistry mechanism
  \item \verb cloud_chemistry : enables the cloud chemistry mechanism
  \item \verb gas_aerosol_exchange : enables exchange processes that occur
                                     between gas and aerosol particles
  \item \verb mode_transfer : enables the transferring particles between modes
  \item \verb nucleation : enables aerosol particle formation processes
  \item \verb coagulation : enables inelastic aerosol collision processes
\end{itemize}

\subsubsection*{Grid Parameters}

This section contains information about the computational grid used by the
Haero driver.

\begin{verbatim}
grid:
  num_columns: 1
  num_levels:  72
\end{verbatim}

There are two required fields:

\begin{itemize}
  \item \verb num_columns : the number of independent atmospheric columns
  \item \verb num_levels : the numer of vertical cells in each column
\end{itemize}

\subsubsection*{Atmospheric Conditions}

In each cell within each column, there exist atmospheric conditions that
provide important parameters---temperature, pressure, relative humidity, etc---
that govern physical processes. The \verb atmosphere  section allows you
to specify one of a handful of simple models for obtaining those parameters.
Each model has its own parameters.

First and foremost, you specify a model with the \verb model  field within
the \verb atmosphere  section. Supported models are:

\begin{itemize}
  \item \verb uniform : a simple atmospheric environment in which columns are
                        assumed to be short in comparison to the height of the
                        atmosphere so that all conditions are uniform
  \item \verb hydrostatic : an atmospheric environment in hydrostatic
                            equilibrium, with the relationship between pressure
                            and temperature defined by an ideal gas law
\end{itemize}

These models are described in detail in \refchapter{driver}. Here we simply
list valid fields for each model. These fields are specified alongside the
\verb model  field in the \verb atmosphere  section. Tables~\ref{tab:uniform_atm}
and~\ref{tab:hydrostatic_atm} list fields for the \verb uniform  and
\verb hydrostatic  models.

\begin{table}[htbp]
\caption{Uniform atmosphere parameters}
\centering
\label{tab:uniform_atm}
\begin{tabular}{ccc}
  \toprule
  Parameter   & Description                  & Units   \\
  \midrule
  \verb mu    & Mean molecular weight of air & kg/mol  \\
  \verb H     & Scaled atmospheric height    & m       \\
  \verb p0    & Pressure                     & Pa      \\
  \verb T0    & Temperature                  & K       \\
  \verb phi0  & Relative humidity            & -       \\
  \verb N0    & Cloud fraction               & -       \\
  \bottomrule
\end{tabular}
\end{table}

\begin{table}[htbp]
\caption{Hydrostatic atmosphere parameters}
\centering
\label{tab:hydrostatic_atm}
\begin{tabular}{ccc}
  \toprule
  Parameter   & Description                  & Units   \\
  \midrule
  \verb mu    & Mean molecular weight of air & kg/mol  \\
  \verb H     & Scaled atmospheric height    & m       \\
  \verb p0    & Pressure                     & Pa      \\
  \verb T0    & Temperature                  & K       \\
  \verb phi0  & Relative humidity            & -       \\
  \verb N0    & Cloud fraction               & -       \\
  \bottomrule
\end{tabular}
\end{table}

\subsubsection*{Initial Conditions}

The \verb initial_conditions section defines the initial state of an aerosol
system. There are subsections for \verb aerosols , for \verb gases , and
for \verb modes  themselves. Alternatively all initial conditions can be read
automatically from a properly-formatted NetCDF file with the following
syntax:

\begin{verbatim}
initial_condition: my_ics_file.nc
\end{verbatim}

The rest of this section discusses how to specify initial conditions for each
aerosol species, gas species, and mode.

The initial state for an aerosol species is specified by its modal mass
fractions, given for every mode occupied by that species. The nodal mass
fraction of an aerosol species is the fraction of the total mass of the mode's
particles occupied by that species.

Meanwhile, the initial state for a gas is its mole fraction---the number of
moles of gas per mole of air.

Finally, an initial mode state is given as a number density: the number of total
aerosol particles in the mode per cubic meter.

In each case, the initial state is specified for an entire column. There are a
few ways of specifying input for a quantity within a column:

\begin{enumerate}
  \item {\bf Specify a uniform quantity with a single number.}
        \begin{verbatim}
          <quantity>: value
        \end{verbatim}
        This is the easiest option for defining a quantity's initial state: it's
        the same everywhere, from the top to the bottom of the column. For
        example, to define a uniform mass fraction of sulfate within a mode:
        \begin{verbatim}
          SO4: 0.3
        \end{verbatim}
  \item {\bf Use a YAML list to specify a quantity's vertical profile.}
        \begin{verbatim}
          <quantity>: [value1, value2, ..., valueN]
        \end{verbatim}
        This option allows you to specify a value of an aerosol, a gas, or a mode
        in each of the cells of a column with a YAML list. For example, if your
        columns have 10 vertical levels, you can specify the mass fraction of
        sulfate within a mode using a list like this:
        \begin{verbatim}
          SO4: [0.28, 0.285, 0.29, 0.295, 0.3, 0.305, 0.31, 0.315, 0.32, 0.325]
        \end{verbatim}
        The size of the list must match the number of vertical levels.
  \item {\bf Refer to a named field in the \verb data  section.}
        \begin{verbatim}
          <quantity>: data: <data_field>
        \end{verbatim}
        This option is similar to specifying a column profile with a YAML list
        of values, except that the list is given a symbolic name within the
        \verb data  section (described below), and the initial condition refers
        to the symbolic name of that list. Suppose you were specifying modal
        mass fractions for sulfate in the \verb accumulation  mode, and the
        mass fraction data is defined in the \verb data  section in the field
        \verb SO4_accum . Then your initial condition field would read:
        \begin{verbatim}
          SO4: data: SO4_accum
        \end{verbatim}
  \item {\bf Refer to a variable in a NetCDF file.}
        \begin{verbatim}
          <quantity>: <filename>: <var_name>
        \end{verbatim}
        If you have a NetCDF file that defines initial conditions for columns,
        you can refer directly to any variable within that file, as long as its
        dimension matches the dimension of the columns in your
        \verb grid  section. For example, to read column mass fractions for
        sulfate within a mode from the \verb SO4_massfrac  variable within a
        NetCDF file named \verb column_ics.nc :
        \begin{verbatim}
          SO4: column_ics.nc: SO4_massfrac
        \end{verbatim}
\end{enumerate}

Here's an example of a simple set of initial conditions in which all aerosols,
gases, and modes are given uniform initial states:

\begin{verbatim}
initial_conditions:
  aerosols:
    accumulation:
      SO4: 0.3
      POA: 0
      SOA: 0.3
      BC: 0
      DST: 0
      NCL: 0.4
    aitken:
      SO4: 0.3
      SOA: 0.3
      NCL: 0.4
    coarse:
      DST: 0
      NCL: 0.4
      SO4: 0.3
      BC: 0
      POA: 0
      SOA: 0.3
    primary_carbon:
      POA: 0
      BC: 1

  gases:
    SO2: 1.e-4
    H2SO4: 1.e-13
    SOAG:  5.e-10

  modes:
    accumulation: 1.e8
    aitken: 1.e9
    coarse: 1.e5
    primary_carbon: 5.e8
\end{verbatim}

In the \verb aerosols  subsection, the modal mass fractions of all species
are defined for each mode. So the \verb aerosols contains mode names as fields.
Each of these mode names contains one or more symbolic names of aerosol species.
Every species and mode referenced in this subsection must be defined earlier
in the file.

In the \verb gases  subsection, the initial state of a gas section is simpler,
since gases don't belong to specific modes. Each of the fields here adopt the
symbolic names of the gases defined in the \verb gases section of the file.

The \verb modes  subsection, the fields are named after the modes defined in
the \verb modes  section.

\subsubsection*{Data Section}

The \verb data  section mentioned above contains fields with values that are
either single numbers or YAML lists. This section is primarily used by fields
in the \verb initial_conditions  section. Here's an example of some valid
entries for a simulation with columns having 10 vertical levels, for data
defined on each of those vertical levels:

\begin{verbatim}
data:
  SO4_accum: [0.28, 0.285, 0.29, 0.295, 0.3, 0.305, 0.31, 0.315, 0.32, 0.325]
  POA_accum: 0
\end{verbatim}

Lists for data defined on vertical level {\em interfaces} must contain one more
element than those for data defined on vertical levels.

\subsubsection*{Perturbations to Initial Conditions}

TBD

\subsubsection*{Chemistry Model}

The Haero driver implements an extremely simple chemistry model: the uniform
production of a gas species at a constant rate. The parameters of this model,
which are the rates at which one or more gas species are uniformly produced
in all grid cells, are defined in the \verb production  subsection of the
\verb chemistry  section, in the following way:

\begin{verbatim}
chemistry:
  production:
    H2S04: 1.0e-16
\end{verbatim}

Here¸ we specify that \verb H2SO4  gas (which must be defined in the
\verb gases  section) gets produced at $ 10^{-16} $ mol gas / mol air / s.

\subsubsection*{Simulation Parameters}

In the \verb simulation  section, you define parameters relevant to your
simulation, as opposed to the system being simulated:

\begin{verbatim}
simulation:
  timestep: 1
  duration: 1800
  output:
    directory: .
    prefix: smoke_test
    frequency: 1
\end{verbatim}

This section contains two fields relevant to timestepping:

\begin{itemize}
  \item \verb timestep : a fixed timestep size used in the simulation (s).
                         This may also be a list of fixed timesteps, in which
                         case the driver will run a self-convergence study
  \item \verb duration : the total duration of the simulation (s)
\end{itemize}

There's also an \verb output  subsection with parameters related to simulation
output. These parameters are defined by the following fields:

\begin{itemize}
  \item \verb directory : the directory in which output files are written
  \item \verb prefix : a prefix for output files for this simulation
  \item \verb frequency : the number of timesteps to advance between writing
                          successive simulation output files.
\end{itemize}

\subsection{Examples}

You can find examples of driver input files in the \verb driver/tests  directory
within the \verb haero  GitHub repository.

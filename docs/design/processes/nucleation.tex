\section{Nucleation}
\labelsection{nucleation}

Aerosol nucleation refers to the formation of new aerosol particles from gas
molecules. Nucleation is an important source of aerosol particles in the
atmosphere, and can have a significant effect on other aerosol-related
processes. As a conversion mechanism, it reduces the population of gas-phase
species and increases aerosol mass and number concentrations.

Aerosol nucleation can also affect the formation of clouds and fog. The newly
formed particles that result from nucleation are generally hydrophilic and not
efficiently scavenged because of their small sizes. These particles can
grow larger and act as nuclei for cloud condensation.

Haero currently offers one process implementation for nucleation: the legacy
MAM4 model, written in Fortran.

\subsection{Nucleation Physics}

In the atmosphere, the gas molecules exist either independently or as small
groups (i.e., as clusters that contain more than one molecule~\cite{seinfeld-2006-acp}).
In general, the concentration of independent gas molecules is much higher than
that of clusters. Therefore, the radius of a cluster is changed mainly due to the collision
between an independent gas molecule and a cluster. If the radius of a cluster is larger
than $r^*$, the radius of the so-called ``critical cluster'' or nucleus, this cluster tends to
grow rapidly and nucleation happens; otherwise, this cluster tends to shrink and nucleation
would not occur. There are some basic types of nucleation:
\begin{itemize}
\item Nucleation could occur homogeneously (in the absence of pre-existing particles) or
heterogeneously (in the presence of pre-existing);
\item Nucleation could occur involving a single species or more than one species.
\end{itemize}
Based on these types, nucleation can be identified as:
\begin{itemize}
\item Type 1: homogeneous nucleation of a single species;
\item Type 2: homogeneous nucleation of two or more species (e.g, water vapor
($\rwater$) and sulfuric acid gas ($\rsag$));
\item Type 3: heterogeneous nucleation of a single species. A good example would be
the formation of water droplet with the pre-existing particles;
\item Type 4: heterogeneous nucleation of two or more species.
\end{itemize}

For all four types, the vapor mixing ratios and aerosol mass and number mixing
ratios are changed during the nucleation process. Their rates of change are
% general equation
\begin{align}
\frac{d q\dsub{m,\specidx}}{dt} &= \frac{d q\dsub{n}}{dt} \cdot \frac{m\dsub{d,\specidx}}{\mw{\specidx}} \,, \label{eqn:aer}\\
\frac{d\vmass{\specidx}}{dt} &= -\frac{d q\dsub{m,\specidx}}{dt} \,, \label{eqn:gas}
\end{align}
where $q\dsub{m,\specidx}$ is the aerosol mass mixing ratio of species $\specidx$
(unit: kmol of species $\specidx$ per kmol of dry air); $q\dsub{n}$ is the aerosol
number mixing ratio (unit: \# of aerosols per kmol of dry air); $q\dsub{v,\specidx}$ is
the mass mixing ratio of gas species $\specidx$ (unit: kmol of species $\specidx$
per kmol of dry air); $m\dsub{d,\specidx}$ is the dry mass of a nucleus containing the
aerosol species $\specidx$ (unit: kg).

% key assumptions
\subsection{MAM4 Nucleation Process}

Though there are four types of aerosol nucleation process in reality, the MAM4
nucleation process focuses only on type 2 nucleation. MAM4 makes several
assumtions (on top of the modal aerosol assumption) to arrive at the
governing equations of its treatment of nucleation:

\begin{assume}
  There is no use a nucleation mode---instead it uses the Aitken mode to capture the smallest
  nucleated aerosol particles.
\end{assume}

Previous research~\cite{kerminen-2002-jas} suggests that the typical size
range of newly formed nuclei are around 1 nm in diameter. Particles of this
size are too small to be represented directly by MAM4's Aitken mode. To capture
the nucleation process, MAM4 breaks the process into two distinct stages.

\begin{assume}
  The nucleation process consists of two stages:
  \begin{enumerate}
    \item Fresh nuclei are formed from the vapors of gas species.
    \item These nuclei grow to a predefined size that falls within the range
      of the Aitken mode. This predefined size is described in
      \refparagraph{nuc:growth_nuclei}.
  \end{enumerate}
  Therefore, nucleation affects only the aerosol number and mass mixing ratios
  in the Aitken mode.
\end{assume}

\begin{assume}
  The mass densities of aerosol sulfuric acid and sulfate are assumed to be the
  same (i.e., $\rm 1770~kg~m^{-3}$);
\end{assume}

\begin{assume}
  To calculate the number concentration of $\rsag$ [\# / cc air], we use the
  average mass mixing ratio of $\rsag$, $c\dsub{n,\sag}$, obtained during the
  condensation process---we use this instead of the mass mixing ratio of $\rsag$
  given at the beginning of the nucleation process. Specifically:

  $$c\dsub{n,\sag} = 6.023 \cdot 10^{20} \vmass{\sag,avg} c\dsub{air}$$

\end{assume}

\begin{assume}
  MAM4 treats only the homogeneous nucleation of $\rsag \mhyphen \rwater$
  (type 2).
\end{assume}

With these assumptions, the governing equations of MAM4's nucleation process
are

\begin{align}
\frac{dq\dsub{n,i}}{dt} &= \frac{10^6 J\dsub{nuc}}{c\dsub{air}} \,, \text{ i = 2 for Aitken mode} \,, \label{eqn:jnuc} \\
\frac{d\amass{\asf}}{dt} &= \frac{dq\dsub{n,i}}{dt} \cdot \frac{m\dsub{d,\asf}}{\mw{\asf}} \,, \text{ i = 2} \,, \label{eqn:amass} \\
\frac{d\vmass{\sag}}{dt} &= -\frac{d\amass{\asf}}{dt} \,, \text{ i = 2} \,, \label{eqn:vmass}
\end{align}

where $J\dsub{nuc}$ [$\rm \# / (cc s)$] is the nucleation rate describing the
net number of clusters that grow past $r^*$ per unit time per $\rm cm^{3}$ of
air. $J\dsub{nuc}$ depends on the available amount of gas species, temperature
and relative humidity; $c\dsub{air}$ [kmol air / $\rm m^3$ air] is the air molar
concentration.

% parameterization
\subsubsection{MAM4 nucleation parameterizations} \labelsubsubsection{paranuc}

In order to solve the governing equations above, it is crucial to solve for the
$J\dsub{nuc}$ in Eq.~\eqref{eqn:jnuc} first. $J\dsub{nuc}$ could be calculated
thermodynamically by the classical nucleation theory. A complete description of
the classical nucleation theory can be found in the Chapter 11 of~\cite{seinfeld-2006-acp}.
However, it is unlikely to introduce the whole classical nucleation theory into MAM
due to its heavy computational burden. Hence parameterizations are implemented
instead to solve for $J\dsub{nuc}$ in MAM. In this documentation, we focus on
the parameterizations of binary nucleation of $\rsag \mhyphen \rwater$ used in
MAM3 and MAM4. Details about the parameterizations in MAM7 can be found
in~\cite{merikanto-2007-jgr}.

First of all, we define the following notations to ease our presentations:
\begin{itemize}
\item RH: clear-sky relative humidity.
%\item $c\dsub{n,\sag}$: number concentration of $\rsag$, unit: number
%of $\rsag$ molecules per $\rm cm^3$ of air.
\item $n\dsub{\sag}$: number of $\rsag$ molecules in a critical cluster.
\item $n\dsub{tot}$: number of molecules in a critical cluster.
\item $J^*$: intermediate nucleation rate (adjusted by each parameterization
in \refparagraph{nuc:binary} to~\refparagraph{nuc:growth_nuclei} to get the final
$J\dsub{nuc}$), unit: $\rm \# \cdot cm^{-3} \cdot s^{-1}$.
\item $C\dsub{sum,\sag}$: sum of mass transfer coefficients of
$\rsag$ over the $I$ modes during the condensation process (unit:
$\rm s^{-1}$). This is a constant for the nucleation process.
\end{itemize}

A \textbf{key assumption} made by all the parameterizations below
is that the newly formed nuclei have a uniform size. With this assumption,
the following nucleation parameterizations are involved in order.
First, RH is bounded within [0.01, 0.99] since the nucleation process in
MAM is only considered for the clear-sky area. In the cloudy area, $\rsag$
would rapidly go into the cloud droplets and the nucleation is unlikely
to happen.

% binary nucleation parameterization by vk2002
\paragraph{Binary nucleation of $\rsag \mhyphen \rwater$}
\labelparagraph{nuc:binary}
The binary parameterization developed by Vehkam\"aki et al.~\citep{vehkamaki-2002-jgr}
is used in MAM, which fits $J^*$ by a polynomial of $T$, $RH$ and $c\dsub{n,\sag}$.
This parameterization has the following limitations of applicable conditions:
%
\begin{itemize}
\item $T \in [230.15, 305.15]$;
\item $RH \in [0.0001, 1.0]$;
\item $c\dsub{n,\sag} \in [10^4, 10^{11}]$;
\item $J^* \in [10^{-7}, 10^{10}]$;
\end{itemize}
%
Only the \textbf{first three} limitations are actually applied in MAM.
$T$ and $RH$ are bounded within the given range. In contrast,
if $c\dsub{n,\sag} \le 10^4$ or $\vmass{\sag,avg} \le 4 \cdot 10^{-16}$
(roughly equivalent condition), nucleation is assumed to not occur
(i.e., $J\dsub{nuc} = 0$ in Eq.~\eqref{eqn:jnuc}) and no further
calculation is done. If the applicable conditions above are satisfied,
this parameterization will compute the following quantities in order:
\begin{enumerate}
\item Mole fraction of $\rsag$ in a critical cluster (i.e., $x^*$). It is
given by:
%
\begin{align}
x^* &= 0.740997 - 0.00266379 T - 0.00349998 \ln c\dsub{\sag}
           + 0.0000504022 T \ln c\dsub{\sag} \nonumber \\
      &+ 0.00201048 \ln RH - 0.000183289 T \ln RH
           + 0.00157407 (\ln RH)^2 \nonumber \\
      &- 0.0000179059 T (\ln RH)^2 + 0.000184403 (\ln RH)^3 \nonumber \\
      &- 1.50345 \cdot 10^{-6} T (\ln RH)^3 \,.
\end{align}
%
\item By knowing $x^*$, the intermediate nucleation rate (i.e., $J^*$) is calculated by:
%
\begin{align}
J^* &= \exp \Big[ a(T,x^*) + b(T,x^*) \ln RH +
         c(T,x^*) (\ln RH)^2 + d(T,x^*) (\ln RH)^3 \nonumber \\
    &+ e(T,x^*) \ln c\dsub{\sag} + f(T,x^*) (\ln RH) \ln c\dsub{\sag} +
         g(T,x^*) (\ln RH)^2 \ln c\dsub{\sag} \nonumber \\
    &+ h(T,x^*) (\ln c\dsub{\sag})^2 + i(T,x^*) (\ln RH)
    (\ln c\dsub{\sag})^2 + j(T,x^*) (\ln c\dsub{\sag})^3 \Big] \,, \label{vk2002_nucrate}
\end{align}
where the coefficients $a(T,x^*) \dots j(T,x^*)$ are given in Appendix~\ref{coeff_nucrate}.
%%
%\begin{align}
%a(T,x^*) &= 0.14309 + 2.21956 T - 0.0273911 T^2 \nonumber \\
%             &+ 0.0000722811 T^3 + \frac{5.91822}{x^*} \,, \\
%b(T,x^*) &= 0.117489 + 0.462532 T - 0.0118059 T^2 \nonumber \\
%             &+ 0.0000404196 T^3 + \frac{15.7963}{x^*} \,, \\
%c(T,x^*) &= -0.215554 - 0.0810269 T + 0.00143581T^2 \nonumber \\
%             &- 4.7758 \cdot 10^{-6} T^3 - \frac{2.91297}{x^*} \,, \\
%d(T,x^*) &= -3.58856 + 0.049508 T - 0.00021382 T^2 \nonumber \\
%             &+ 3.10801 \cdot 10^{-7} T^3 - \frac{0.0293333}{x^*} \,, \\
%e(T,x^*) &= 1.14598 - 0.600796 T + 0.00864245 T^2 \nonumber \\
%             &- 0.0000228947 T^3 - \frac{8.44985}{x^*} \,, \\
%f(T,x^*) &= 2.15855 + 0.0808121 T - 0.000407382 T^2 \nonumber \\
%             &- 4.01957 \cdot 10^{-7} T^3 + \frac{0.721326}{x^*} \,, \\
%g(T,x^*) &= 1.6241 - 0.0160106 T + 0.0000377124 T^2 \nonumber \\
%             &+ 3.21794 \cdot 10^{-8} T^3 - \frac{0.0113255}{x^*} \,, \\
%h(T,x^*) &= 9.71682 - 0.115048 T + 0.000157098 T^2 \nonumber \\
%             &+ 4.00914 \cdot 10^{-7} T^3 + \frac{0.71186}{x^*} \,, \\
%i(T,x^*) &= -1.05611 + 0.00903378 T - 0.0000198417 T^2 \nonumber \\
%            &+ 2.46048 \cdot 10^{-8} T^3 - \frac{0.0579087}{x^*} \,, \\
%j(T,x^*) &= -0.148712 + 0.00283508 T - 9.24619 \cdot 10^{-6} T^2 \nonumber \\
%            &+ 5.00427 \cdot 10^{-9} T^3 - \frac{0.0127081}{x^*} \,.
%\end{align}
%%
\item By knowing $x^*$, the total number of molecules in a critical
cluster (i.e., $n\dsub{tot}$) is calculated by:
%
\begin{align}
n\dsub{tot} &= \exp \Big[ A(T,x^*) + B(T,x^*) \ln RH +
                  C(T,x^*) (\ln RH)^2 + D(T,x^*) (\ln RH)^3 \nonumber \\
           &+ E(T,x^*) \ln c\dsub{\sag} + F(T,x^*) \ln RH \ln c\dsub{\sag} +
                G(T,x^*) (\ln RH)^2 \ln c\dsub{\sag} \nonumber \\
           &+ H(T,x^*) (\ln c\dsub{\sag})^2 +
                 I(T,x^*) \ln RH (\ln c\dsub{\sag})^2 +
                 J(T,x^*) (\ln c\dsub{\sag})^3 \Big] \,, \label{vk2002_ntot}
\end{align}
%
where the coefficients $A(T,x^*) \dots J(T,x^*)$ are given in Appendix~\ref{coeff_ntot}.
%%
%\begin{align}
%A(T,x^*) &= -0.00295413 - 0.0976834 T + 0.00102485 T^2 \nonumber \\
%             &- 2.18646 \cdot 10^{-6} T^3 - \frac{0.101717}{x^*} \\
%B(T,x^*) &= -0.00205064 - 0.00758504 T + 0.000192654 T^2 \nonumber \\
%              &- 6.7043 \cdot 10^{-7} T^3 - \frac{0.255774}{x^*} \\
%C(T,x^*) &= 0.00322308 + 0.000852637 T - 0.0000154757 T^2 \nonumber \\
%              &+ 5.66661 \cdot 10^{-8} T^3 + \frac{0.0338444}{x^*} \\
%D(T,x^*) &= 0.0474323 - 0.000625104 T + 2.65066 \cdot 10^{-6} T^2 \nonumber \\
%              &- 3.67471 \cdot 10^{-9} T^3 - \frac{0.000267251}{x^*} \\
%E(T,x^*) &= -0.0125211 + 0.00580655 T - 0.000101674 T^2 \nonumber \\
%             &+ 2.88195 \cdot 10^{-7} T^3 + \frac{0.0942243}{x^*} \\
%F(T,x^*) &= -0.038546 - 0.000672316 T + 2.60288 \cdot 10^{-6} T^2 \nonumber \\
%              &+ 1.19416 \cdot 10^{-8} T^3 - \frac{0.00851515}{x^*} \\
%G(T,x^*) &= -0.0183749 + 0.000172072 T - 3.71766 \cdot 10^{-7} T^2 \nonumber \\
%              &- 5.14875 \cdot 10^{-10} T^3 + \frac{0.00026866}{x^*} \\
%H(T,x^*) &= -0.0619974 + 0.000906958 T - 9.11728 \cdot 10^{-7} T^2 \nonumber \\
%              &- 5.36796 \cdot 10^{-9} T^3 - \frac{0.00774234}{x^*} \\
%I(T,x^*) &= 0.0121827 - 0.00010665 T + 2.5346 \cdot 10^{-7} T^2 \nonumber \\
%             &- 3.63519 \cdot 10^{-10} T^3 + \frac{0.000610065}{x^*} \\
%J(T,x^*) &= 0.000320184 - 0.0000174762 T + 6.06504 \cdot 10^{-8} T^2 \nonumber \\
%             &- 1.42177 \cdot 10^{-11} T^3 + \frac{0.000135751}{x^*}
%\end{align}
%%
\item By knowing $x^*$ and $n\dsub{tot}$, the total number of $\rsag$
molecules in a critical cluster (i.e., $n\dsub{\sag}$) is calculated by:
%
\begin{equation}
n\dsub{\sag} = n\dsub{tot} x^* \,.
\end{equation}
%
\item By knowing $x^*$ and $n\dsub{tot}$, the radius of critical cluster
$r^*$ (unit: nm) is calculated by:
%
\begin{equation}
r^* = \exp \left[ -1.6524245 + 0.42316402 x^* +
        0.3346648 \ln n\dsub{tot} \right] \,.
\end{equation}
%
\end{enumerate}

% binary nucleation parameterization by wang2009
\paragraph{Nucleation within planet boundary layer} \labelparagraph{nuc:pbl}
If the nucleation occurs within the planet boundary layer (PBL) or below
100-meter height, an additional parameterization with an empirical
first-order estimation is used to adjust $J^*$ if necessary
~\cite{sihto-2006-acp,wang-2009-acp}:
\begin{equation}
J\dsub{PBL} = 10^{-6}~c\dsub{\sag} \,.
\end{equation}
If $J\dsub{PBL} > J^*$, and \textbf{assuming} that
1) fresh nuclei only consist of $\rsag$ and
2) the initial diameter of nuclei is 1 nm,
the following adjustments are done:
\begin{align}
J^* &= J\dsub{PBL} \,, \\
r^* &= 0.5~nm \,, \\
n\dsub{\sag} &= \frac{6.023 \times 10^{23} \pi (1~nm)^3
                           \rho\dsub{\sag}}{6 \mw{\sag}} \,, \\
n\dsub{tot} &= n\dsub{\sag} \,.
\end{align}
Otherwise $J^*$, $n\dsub{tot}$ and $n\dsub{\sag}$ remain the same values
calculated by the first parameterization in \refparagraph{nuc:binary}.

% parameterization of nuclei growth
\paragraph{Nuclei growth} \labelparagraph{nuc:growth_nuclei}
If $J^* < 10^{-6}$ based on the previous calculations, nucleation is
\textbf{assumed} to not occur (i.e., $J\dsub{nuc} = 0$ in Eq.~\eqref{eqn:jnuc}).
Otherwise, the following quantities are calculated:
%
\begin{align}
RH &\in [0.1, 0.95] \,, \\
f\dsub{v} &= 1 - \frac{0.56}{\ln RH} \,, \\
V\dsub{d,nuc} &= \frac{n\dsub{\sag} \mw{\asf}}
                          {6.023 \cdot 10^{26} \rho\dsub{\sag}} \,, \\
D\dsub{d,nuc} &= \left( \frac{6 V\dsub{d,nuc}}{\pi} \right)^{\frac{1}{3}} \,,
\end{align}
%
where $f\dsub{v}$ is the ratio of wet volume over dry volume of a particle,
using a simple K\"ohler approximation for $\rm NH\dsub{4}HSO\dsub{4}$
(This approximation follows the K\"ohler theory used for the aerosol water
uptake in MAM but neglects the surface curvature effect. The composition
is assumed as pure ammonium bisulfate with hygroscopicity = 0.56);
$V\dsub{d,nuc}$ and $D\dsub{d,nuc}$ are the dry volume (unit: $\rm m^3$)
and diameter (unit: m) of a nucleus estimated from the two parameterizations
above. Take the predefined lowest, mean and highest values of particle size
(unit: m) in Aitken mode and we could calculate the two quantities below:
\begin{align}
D\dsub{p,lo} &= \exp \left[ 0.67 \ln (8.7 \cdot 10^{-9}) +
                          0.33 \ln (26 \cdot 10^{-9}) \right] \,, \\   %% about 12.5 nm
D\dsub{p,hi} &= 52 \cdot 10^{-9} \,.
\end{align}
\textcolor{red}{Jian: Dick said $D\dsub{p,lo}$=8.7 nm might make more sense.}

The following two situations are treated separately:
\begin{itemize}
\item If $D\dsub{d,nuc} > D\dsub{p,lo}$, MAM sets $J\dsub{nuc} = J^*$.
% KK2002 scheme
\item If $D\dsub{d,nuc} \leq D\dsub{p,lo}$, these nuclei are too small
in size to be well represented by the size distribution function of
Aitken mode. Thus MAM uses the following parameterization~\cite{kerminen-2002-jas}
to consider the growth of nuclei to a larger size, which comes with some
key assumptions:
\begin{itemize}
%\item The nuclei produced by the nucleation process have a uniform size;
\item The only important sink for the nuclei is their coagulation with the
pre-existing particles; That is, the self-coagulation of fresh nuclei is not
treated in MAM.
\item The only important source for the growth of nuclei is the condensation
of vapors and the condensation rate is assumed to be constant;
\item The population of pre-existing particles and the concentration of
condensible vapors remain unchanged;
\item The condensable vapors responsible for nuclei growth are non-volatile.
\end{itemize}
Using this parameterization, MAM then adjusts the $J^*$ based on
1) the increase of the size of nuclei due to the condensation of $\rsag$,
and 2) the decrease of the number concentration of nuclei due to the
coagulation with pre-existing particles. The following quantities are
calculated in order:
%
\begin{align}
v\dsub{\sag} &= 14.7 \sqrt{T} \,, \\
\rho\dsub{nuc} &= \frac{\rho\dsub{\sag}}{f\dsub{v}} \,, \\
GR &= \frac{3.0 \cdot 10^{-9} v\dsub{\sag} \mw{\sag} c\dsub{n,\sag}}
                   {\rho\dsub{nuc}} \,, \\
D\dsub{nuc,ini} &= \max (2r^*, 1) \,, \\
D\dsub{nuc,fin} &= 10^9 D\dsub{p,lo} f\dsub{v}^{\frac{1}{3}} \,, \\
\gamma &= 0.23 D\dsub{nuc,int}^{0.2}
                    (\frac{D\dsub{nuc,fin}}{3})^{0.075}
                    (\frac{\rho\dsub{nuc}}{1000})^{-0.33}
                    (\frac{T}{293})^{-0.75} \,, \label{eq:gamma} \\
\mathbb{D}\dsub{g,\sag} &= \frac{6.7037 \cdot 10^{-9} T^{0.75}}
                                             {c\dsub{air}} \,, \\
CS^\prime &= \frac{C\dsub{sum,\sag}}
                  {4 \pi \mathbb{D}\dsub{g,\sag} \alpha\dsub{\sag}} \,, \label{eq:cs_prime} \\
\eta &= \frac{\gamma CS^\prime}{GR} \,, \\
J\dsub{nuc} &= J^* \exp \left( \frac{\eta}{D\dsub{nuc,fin}} -
                         \frac{\eta}{D\dsub{nuc,ini}} \right) \,,
\end{align}
%
where
\begin{itemize}
\item $v\dsub{\sag}$ is the approximated mean molecular speed of
$\rsag$ (unit: $\rm m~s^{-1}$);
\item $\rho\dsub{nuc}$ is the density of nuclei (unit: kg per $\rm m^3$
of nuclei) after considering the aerosol water uptake by sulfate;
\item $GR$ is the nuclei growth rate (unit: $\rm m~s^{-1}$), which
is a constant based on the assumption that nuclei growth rate by
condensation is constant;
\item $D\dsub{nuc,ini}$ and $D\dsub{nuc,fin}$ are the wet diameters
(unit: nm) before and after nuclei growth;
\item $\mathbb{D}\dsub{g,\sag}$ is the approximated gas diffusivity
of $\rsag$ (unit: $\rm m^2~s^{-1}$);
\item $CS^\prime$ is called the ``condensation sink'', which is
a constant based on the assumption that the population of
pre-existing particles does not change. The formular of $CS^\prime$
used in MAM comes from the condensation equation of $\rsag$ gas and
is different from that in~\cite{kerminen-2002-jas}. Dick commented
that this would lead to minor difference.
% and the detailed derivation can be found in the Appendix~\ref{cs_prime}.
\end{itemize}
Note that compared with the Eq. (22) in~\cite{kerminen-2002-jas},
Dick also dropped a term in Eq.~\eqref{eq:gamma}, which was
thought to be close to 1.
\end{itemize}

\subsubsection{MAM4 nucleation numerics}

After MAM calculates $J\dsub{nuc}$ at $t = t\dsub{0}$ (i.e., $J\dsub{nuc}(t\dsub{0})$)
using the parameterizations in \refsubsubsection{paranuc}, the increase of
$\aitmass{\asf}$ (i.e., $\Delta \aitmass{\asf}$) due to nucleation can be
calculated by:
%
\begin{align}
\Delta \aitmass{\asf} &= \frac{10^6 J\dsub{nuc}(t\dsub{0}) \Delta t\dsub{nuc} m\dsub{d,\asf}}
                          {c\dsub{air}\mw{\asf}} \,, \\
m\dsub{d,\asf} &= \begin{cases}
\rho\dsub{\sag} \frac{\pi}{6}V\dsub{p,hi}^3 \,, \, \text{if $D\dsub{d,nuc} \ge D\dsub{p,hi}$} \,, \\
\rho\dsub{\sag} \frac{\pi}{6}V\dsub{d,nuc}^3 \,, \, \text{if $D\dsub{p,hi} > D\dsub{d,nuc} > D\dsub{p,lo}$} \,, \\
\rho\dsub{\sag} \frac{\pi}{6}V\dsub{p,lo}^3 \,, \, \text{otherwise} \,,
\end{cases}
\end{align}
%
where $\Delta t\dsub{nuc}$ = $t\dsub{1}$ - $t\dsub{0}$ = $\Delta t\dsub{phys}$
(See ``MAM\_Basics'' documentation for the value of $\Delta t\dsub{phys}$).
Since $\Delta \aitmass{\asf}$ could not exceed the available amount of
$\vmass{\sag}(t\dsub{0})$, the following fraction is calculated:
\begin{equation}
f^* = \begin{cases}
\frac{\vmass{\sag}(t\dsub{0})}{\Delta \aitmass{\asf}}, \, \text{if $\Delta \aitmass{\asf} > \vmass{\sag}(t\dsub{0})$} \,, \\
1, \text{otherwise.} \\
\end{cases}
\end{equation}
If $J\dsub{nuc} f^* < 10^{-18}$, nucleation is assumed to not occur (i.e.,
$J\dsub{nuc} = 0$); otherwise:
%
\begin{align}
\Delta \aitmass{\asf} &= \min ( 0.9999\vmass{\sag}(t\dsub0), f^* \Delta \aitmass{\asf} ) \,, \\
\Delta \vmass{\sag} &= - \Delta \aitmass{\asf} \,, \\
\Delta q\dsub{n,2} &= \frac{\Delta \aitmass{\asf} \mw{\asf}}{m\dsub{d,\asf}} \,.
\end{align}
%
If $\frac{\Delta q\dsub{n,2}}{\Delta t\dsub{nuc}} < 100$,
nucleation is assumed to not occur; otherwise:
\begin{itemize}
\item If $\frac{\Delta \aitmass{\asf} \mw{\asf}}{\Delta q\dsub{n,2}} <
\frac{\pi}{6} \rho\dsub{\asf} D\dsub{p,lo}^3$, MAM increases the size of nuclei by
reducing the number of nuclei, that is, $\Delta q\dsub{n,2} =
\frac{\Delta \aitmass{\asf} \mw{\asf}}{\frac{\pi}{6} \rho\dsub{\asf} D\dsub{p,lo}^3}$;
\item If $\frac{\Delta \aitmass{\asf} \mw{\asf}}{\Delta q\dsub{n,2}} >
\frac{\pi}{6} \rho\dsub{\asf} D\dsub{p,hi}^3$, MAM decreases the size of nuclei by
reducing the mass of nuclei, that is,
$\Delta \aitmass{\asf} = \frac{\frac{\pi}{6} \rho\dsub{\asf} D\dsub{p,hi}^3 \Delta q\dsub{n,2}}{\mw{\asf}}$.
\end{itemize}
Dick commented that 1) the first ``if'' condition revealed the fact that
there was not enough $\rsag$ gas for all the nuclei to grow to the size
$D\dsub{p,lo}$. Thus Dick arbitrarily reduced the number of nuclei to
increase the size of nuclei; 2) the second ``if'' condition will never
happen, so it is just a ``sanity check''. After the adjustment above,
MAM solves the governing equations~\eqref{eqn:jnuc} to~\eqref{eqn:vmass} as:
\begin{align}
q\dsub{n,2} (t\dsub{1}) &= q\dsub{n,2} (t\dsub{0}) + \Delta q\dsub{n,2} \,, \\
\aitmass{\asf} (t\dsub{1}) &= \aitmass{\asf} (t\dsub{0}) + \Delta \aitmass{\asf} \,, \\
%\Delta q\dsub{\sag} &= \min \left( \Delta \aitmass{\asf}, \vmass{\sag} (t\dsub{0}) \right) \,, \\
\Delta q\dsub{\sag} &= -\Delta \aitmass{\asf} \,, \\
\vmass{\sag} (t\dsub{1}) &= \vmass{\sag} (t\dsub{0}) + \Delta q\dsub{\sag} \,.
\end{align}

\subsubsection{Additional MAM4 nucleation diagnostics} \label{output}
%%
The basic outputs from the nucleation process include:
\begin{enumerate}
\item $J^*$: The nucleation rate after the first and second (if applicable) parameterizations;
\item $q\dsub{n,2}$: The number mixing ratio of $\rasf$ in the Aitken mode at $t = t\dsub1$;
\item $\vmass{\sag} (t\dsub1)$: The mass mixing ratio of $\rsag$ at $t = t\dsub1$;
\item $\aitmass{\asf} (t\dsub1)$: The mass mixing ratio of $\rasf$ in the Aitken mode at $t = t\dsub1$.
\end{enumerate}


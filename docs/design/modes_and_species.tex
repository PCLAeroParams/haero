\section{Modes, Species, and Chemical Reactions}
\labelsection{modes_and_species}

In this section, we introduce data types in Haero that are used to define a
physical system involving aerosols. These data types don't hold state
information---they only specify the modes and species present in a system,
and the chemical reactions that affect their populations.

Haero is written primarily in C++, so we describe C++ data structures in
this section. However, Haero also provides a Fortran interface that can be
used by scientists to develop and test new parametrizations. Both the C++
and Fortran application programming interfaces (APIs) are described in
detail in \refappendix{api}.

\subsection{The Mode Data Type}\labelsection{mode_data_type}

We've seen how the dynamics of aerosols can be represented mathematically
by evolution equations for moments of modal distribution functions. Modes
simplify the description of aerosol particles in terms of their size: instead of
representing a population of particles with a distribution function
$n(V_p, \vec{x}, t)$ that varies continuously with the size of the particle, we
introduced $M$ discrete modes and declared that these modes partition the
population of aerosol particles in the sense of the modal assumption as given
by \refeq{modal_n}.

The essential information in a mode is the range of particle sizes it
encompasses, $[D_{\min}, D_{\max}]$, and its geometric standard deviation,
$\sigma_g$. In Haero's C and C++ interfaces,
we represent an aerosol mode with the \verb|Mode| struct:

\begin{verbatim}
struct Mode {
  std::string name;  // a unique identifier for the mode
  Real min_diameter; // the mode's minimum particle diameter
  Real max_diameter; // the mode's maximum particle diameter
  Real mean_std_dev; // the geometric mean standard deviation for the mode
}
\end{verbatim}

In principle, a Haero calculation can support any number of modes, but care
must be taken to ensure that the modal assumptions remain valid, and that
the parametrizations selected can accommodate the given modes.

\subsection{The Species Data Type}\labelsection{species_data_type}

{\bf WARNING: This section is under construction!}

Each aerosol mode consists of one or more particle species. Additionally,
gas particles also come in different species. Both aerosol particles and
gas particles have physical properties that are described by the
\verb Species  data type.

A particle species is a specifically-identified molecular assembly with
a number of relevant physical properties. The fundamental description of a
species includes

\begin{itemize}
  \item a symbolic name (e.g. \verb SO4 , for sulfate)
  \item a descriptive name (e.g. \verb "sulfate" )
  \item its elemental composition (how many atoms of each type it possesses)
  \item its net electrical charge (in units of the elementary charge $|e|$)
\end{itemize}

We represent this information in the following way:

\begin{verbatim}
struct Species {
  std::string name;                       // full species name
  std::string symbol;                     // abbreviated symbolic name
  std::map<std::string, int> composition; // elemental make-up of species
  int charge;                             // net electrical charge
};
\end{verbatim}

The \verb composition  field in the \verb Species  struct maps the name of
a chemical element to the number of atoms of that type that appear within
the species.

\subsection{Chemistry Data Types}\labelsection{chem_data_types}

TBD: we currently don't have a candidate for a chemical mechanism, though
we are considering Cantera.


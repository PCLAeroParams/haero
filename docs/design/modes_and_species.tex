\section{Modes, Species, and Chemical Reactions}
\labelsection{modes_and_species}

In this section, we introduce data types in MAM that are used to define a
physical system involving aerosols. These data types don't hold state
information--they only specify the modes and species present in a system,
and the chemical reactions that affect their populations.

\subsection{The Mode Data Type}\labelsection{mode_data_type}

We've seen how the dynamics of aerosols can be represented mathematically
by evolution equations for moments of modal distribution functions. Modes
simplify the description of aerosol particles in terms of their size: instead of
representing a population of particles with a distribution function
$n(V_p, \vec{x}, t)$ that varies continuously with the size of the particle, we
introduced $M$ discrete modes and declared that these modes partition the
population of aerosol particles in the sense of the modal assumption as given
by \refeq{modal_n}.

The essential information in a mode is the range of particle sizes it
encompasses, $[D_{\min}, D_{\max}]$, and its geometric standard deviation,
$\sigma_g$. In MAM's C and C++ interfaces,
we represent an aerosol mode with the \verb|Mode| struct:

\begin{verbatim}
struct Mode {
  char name[32]; // A string identifier for the mode
  int index;     // The mode's index
  Real D_min;    // The mode's minimum particle diameter
  Real D_max;    // The mode's maximum particle diameter
  Real sigma;    // The geometric standard deviation for the mode
}
\end{verbatim}

Note that we provide a unique name that identifies the mode and a unique
index for that mode within the context of a simulation, for bookkeeping
purposes. In Fortran, we provide an equivalent derived type:

\begin{verbatim}
type Mode
  integer           :: index
  character(len=32) :: name
  real(wp)          :: D_min
  real(wp)          :: D_max
  real(wp)          :: sigma
end type Mode
\end{verbatim}

\subsection{The Species Data Type}\labelsection{species_data_type}

{\bf WARNING: This section is under construction!}

Each aerosol mode consists of one or more particle species.

A particle species is a specifically-identified molecular assembly with
a number of relevant physical properties. The fundamental description of a
species includes

\begin{itemize}
  \item a descriptive name and a corresponding unique index, for bookkeeping
  \item its elemental composition (how many of which atoms it possesses)
  \item its net electrical charge (in units of the elementary charge $e$)
  \item an appropriate equation of state for computing thermodynamic quantities
\end{itemize}

Because the composition of a species can involve an arbitrary number of
elements, we use an interface (or ``object") to represent a species instead of
a struct. In C++, a \verb|Species| is a class with methods for defining and
querying its composition. In C and Fortran, a \verb|species_t| (the equivalent
type) is an ``opaque'' type with associated functions/subroutines for defining
and querying its composition.

\subsubsection*{Constructor}

You construct a species object by calling a {\bf constructor}---a
function that creates a new species for use in MAM.

\begin{verbatim}
  ...
  type(species_t) species
  character(len=32) name
  ...
  name = "O3"
  species = species_new(name, composition, charge, eos)

\end{verbatim}

\subsubsection*{Definition Functions}

\subsubsection*{Query Functions}

\subsubsection*{Destructor}

At the end of a simulation, MAM destroys all of its data, freeing up any
resources used. When a species object is destroyed its {\bf destructor }
function is called, which performs any necessary deallocation.

Typically, you don't need to call a destructor outside of a testing environment.
But for completeness, here's how you call the destructor for a species object:

\begin{verbatim}
  call species_free(species)
\end{verbatim}

\subsection{Chemistry Data Types}\labelsection{chem_data_types}


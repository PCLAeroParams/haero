\section{Introduction}
\labelsection{intro}

Aerosols and aerosol-cloud interactions remain a great source of uncertainty
in global climate models. The Modal Aerosol Model (MAM) described in this
document attempts to help researchers better understand the relevant physical
processes and their contributions to the global climate.

Any model must define abstractions and make assumptions in order to produce
quantitative answers to scientific questions. The abstractions and assumptions
for MAM live in this document. We also describe MAM's software
components---its data structures and programming interface---which are designed
to match these abstractions. In particular, we aim to provide an interface that
allows a scientist to write code that resembles statements made in technical
conversations. We hope that this close correspondence between code and
the related scientific discourse can allow people with different areas of
expertise to contribute to the development of MAM.

In \refsection{equations}, we very briefly outline the governing equations of
aerosol dynamics from a mathematical perspective. We begin from basic transport
equations formulated in terms of aerosol size distribution functions. We also
describe the assumptions that enter the model at the level of continuous
mathematics (whose approximation errors are present even in analytic
solutions!).

Because the coupling of aerosol-related processes is still an active area of
research, MAM's software interface focuses on defining specific physical and
mathematical entities, providing a set of elementary building blocks from which
more elaborate models can be constructed. In particular, the interface defines
representations for:

\begin{itemize}
  \item {\bf Modes}: statistically-informed groupings of aerosol particles by
        size
  \item {\bf Species}: aerosol molecules of interest, each of which belongs to a
        single mode and is tracked by mass and number.
\end{itemize}

These two entities represent the aerosols in MAM, and provide {\bf metadata} needed to
perform calculations on {\bf state data}, which is stored in multi-dimensional
arrays of real numbers. Modes and species and their associated data structures
are explored in \refsection{modes_and_species}.

The modes and species present in the atmosphere don't uniquely specify the
parameters of a simulation, however. These species interact chemically with each
other, and with other species and also with radiation.  Additionally, there are
numerical parameters that define grid parameters, choices of algorithms, and
other non-physical considerations. This information lives within an opaque
object called a {\bf simulation context}. There's one such context for each
simulation, and MAM uses the context to make decisions about how to numerically
compute the evolution of aerosols, and their effects on a system. The simulation
context and its interface is explained in \refsection{contexts}.

Finally, given a system of modes, species, chemical reactions, etc, MAM's
application programming interface (API) provides functions (sometimes called
subroutines) that evaluate aerosol {\bf tendencies}---instantaneous temporal
rates of change in their mass and number---as well as contributions to energy
equations and coupling terms used by other physical processes. Functions are
provided for the C, C++, and Fortran programming languages. These functions are
described in detail in \refsection{tendencies}.

A comprehensive description of the API appears in \refappendix{api}.

